\documentclass[11pt]{article}

%%%%%%%%%%%%
% Packages %
%%%%%%%%%%%%
\hyphenpenalty=10000
\usepackage{tocloft}
\renewcommand\cftsecleader{\cftdotfill{\cftdotsep}}

\usepackage{xcolor}
\usepackage{hyperref}
\usepackage{epstopdf}
\usepackage{braket}
\usepackage{upgreek}
\usepackage{caption}
\usepackage{booktabs}
\usepackage{subcaption}
\usepackage{amssymb,latexsym,amsmath,gensymb}
\usepackage{latexsym}
\usepackage{graphicx}
\usepackage{float}
\usepackage{enumitem}
\usepackage{pdflscape}
\usepackage{url}
\usepackage{tikz, calc}
\usetikzlibrary{shapes.geometric, arrows, calc}
\tikzstyle{norm} = [rectangle, rounded corners, minimum width=2cm, minimum height=1cm,text centered, draw=black]
\tikzstyle{arrow} = [thick, ->, >=stealth]

\providecommand{\e}[1]{\ensuremath{\times 10^{#1}}} 
\providecommand{\mb}[1]{\mathbf{#1}}
\providecommand{\mh}[1]{\mathbf{\hat{#1}}}
\providecommand{\bs}[1]{\boldsymbol{#1}} 
\providecommand{\intinf}{\int_{-\infty}^{\infty}}
\providecommand{\fig}[4]{
  % filename, width, caption, label
\begin{figure}[h]
 \captionsetup{width=1.0\linewidth}
 \centering
 \includegraphics[width = #2\textwidth]{#1}
 \caption{#3}
 \label{fig:#4}
\end{figure}
}

\newcommand{\tensor}[1]{\overset{\text{\tiny$\leftrightarrow$}}{\mb{#1}}}
\newcommand{\tunderbrace}[2]{\underbrace{#1}_{\textstyle#2}}
\providecommand{\figs}[7]{
  % filename1, filename2, caption1, caption2, label1, label2, shift
\begin{figure}[H]
\centering
\begin{minipage}[b]{.4\textwidth}
  \centering
  \includegraphics[width=1.0\linewidth]{#1}
  \captionsetup{justification=justified, singlelinecheck=true}
  \caption{#3}
  \label{fig:#5}
\end{minipage}
\hspace{2em}
\begin{minipage}[b]{.4\textwidth}
  \centering
  \includegraphics[width=1.0\linewidth]{#2}
  \vspace{#7em}
  \captionsetup{justification=justified}
  \caption{#4}
  \label{fig:#6}
\end{minipage}
\end{figure}
}
\makeatletter

\providecommand{\code}[1]{
\begin{center}
\lstinputlisting{#1}
\end{center}
}

%%%%%%%%%%%
% Spacing %
%%%%%%%%%%%
% Margins
\usepackage[
top    = 1.5cm,
bottom = 1.5cm,
left   = 1.5cm,
right  = 1.5cm]{geometry}

% Indents, paragraph space
\usepackage{parskip} 

% Section spacing
\usepackage{titlesec}
\titlespacing*{\title}
{0pt}{0ex}{0ex}
\titlespacing*{\section}
{0pt}{0ex}{0ex}
\titlespacing*{\subsection}
{0pt}{0ex}{0ex}
\titlespacing*{\subsubsection}
{0pt}{0ex}{0ex}

% Line spacing
\linespread{1.1}

%%%%%%%%%%%%
% Document %
%%%%%%%%%%%%
\begin{document}
\title{\vspace{-2.5em}Single Molecule Fluorescence Microscopy Update\vspace{-1em}}
\author{Talon Chandler}% and Patrick La Rivi\`ere}
\date{\vspace{-1em}March 3, 2017\vspace{-1em}}
\maketitle

\section{Current Model and Limitations}
In the previous note set, we developed the following forward model
\begin{align}
  I_{\text{img}}(\mb{r''}) = \left|\mathcal{F}_{3D}\left\{
    \mb{\tilde{O}}_{\text{obj}}\tensor{\mathbf{G}}_{FF}\tensor{\bs{\alpha}}\mb{E}_{\text{in}}
  \right\}
  \right|^2.\label{eq:5}
\end{align}

Rudolph and Shalin identified several limitations of the model:
\begin{itemize}
\item The model needlessly tracks phase from start to finish. Rudolph and Shalin
  correctly pointed out that phase is irrelevant on the illumination side
  because we measure the intensity over long time periods compared to the period
  of the electric fields.
\item The model only considers monochromatic illumination by a single plane wave
  and focused laser illumination, not K\"{o}hler illumination by a broadband
  source.
\item The model only considers excitation of a single fluorophore. 
\end{itemize}
These notes address these limitations. 

\subsection{Single Fluorophore, Monochromatic Plane Wave}
Consider a monochromatic plane wave with polarization axis $\mb{E}$
incident on a single fluorophore with absorption dipole moment axis
$\hat{\bs{\mu}}_{\text{abs}}$ and emission dipole moment
$\hat{\bs{\mu}}_{\text{em}}$. The incident plane wave induces a dipole moment given by
\begin{align}
  \bs{\mu}_{\text{ind}} \propto \hat{\bs{\mu}}_{\text{em}} \left[\hat{\bs{\mu}}_{\text{abs}}\cdot \mb{E}\right].
\end{align}

\subsection{Many Fluorophores, Monochromatic Plane Wave}
Now consider $N$ fluorophores in the specimen volume with absorption dipole
moments $\hat{\bs{\mu}}_{\text{abs},i}$, emission dipole moments
$\hat{\bs{\mu}}_{\text{em},i}$, and positions $\mb{r}_i$. We define the
\textit{emission dipole moment map} as
$\bs{\mu}_{\text{em}}(\mb{r}) \equiv \sum_{i=0}^N \hat{\bs{\mu}}_{\text{em},i}
\delta (\mb{r} - \mb{r}_i)$. Similarly, we define the \textit{absorption dipole
  moment map} as
$\bs{\mu}_{\text{abs}}(\mb{r}) \equiv \sum_{i=0}^N \hat{\bs{\mu}}_{\text{abs},i}
\delta (\mb{r} - \mb{r}_i)$. When we illuminate the fluorophores with a
monochromatic plane wave we create the \textit{induced dipole map} given by
\begin{align}
    \bs{\mu}_{\text{ind}}(\mb{r}) \propto \bs{\mu}_{\text{em}}(\mb{r}) \left[\bs{\mu}_{\text{abs}}(\mb{r})\cdot \mb{E}\right].
\end{align}

\subsection{Many Fluorophores,  Focused Laser Illumination}
Under focused laser illumination the electric field in the specimen volume
is position dependent. If $\mb{E}(\mb{r})$ is the real 3D electric field
direction at every point in the specimen volume, then the induced dipole map is given
by
\begin{align}
    \bs{\mu}_{\text{ind}}(\mb{r}) \propto \bs{\mu}_{\text{em}}(\mb{r}) \left[\bs{\mu}_{\text{abs}}(\mb{r})\cdot \mb{E}(\mb{r})\right].
\end{align}

\subsection{Many Fluorophores, Broadband Plane Wave}
If we illuminate a fluorophore with a broadband plane wave and measure the
induced dipole moment over a period of time much longer than the coherence time,
then the effective induced dipole moment will be the sum of the induced dipole
moments created by each frequency component. In other words, each frequency
component acts independently on the fluorophore to create an induced dipole
moment.

If we want to compare illumination sources with different spectra, we will need
to consider the \textit{excitation efficiency}, $\eta(\omega)$, of each
frequency $\omega$. For a two level system, $\eta(\omega) = 1$ at the resonance
frequency and drops to 0 far from the resonance frequency following a Lorentzian
function. $\eta(\omega)$ is a more complicated function for real
fluorophores. The induced dipole map is given by
\begin{align}
    \bs{\mu}_{\text{ind}}(\mb{r}) \propto \intinf d\omega\ \eta(\omega) \bs{\mu}_{\text{em}}(\mb{r}) \left[\bs{\mu}_{\text{abs}}(\mb{r})\cdot \mb{E}(\omega)\right].
\end{align}

\subsection{Many Fluorophores, K\"{o}hler Illumination}
If we illuminate a fluorophore with two plane waves with a random phase
difference traveling in different directions and measure the induced dipole over
a long time compared with the coherence time of the two waves, the induced
dipole is the sum of the induced dipole created by each plane wave
independently. In other words, each plane wave creates an induced dipole
independently. (Thank you Rudolph and Shalin for steering me in this direction).

Under K\"{o}hler illumination with a polarizer in the back focal plane, each
paraxial point in the back focal plane of the condenser creates a polarized
plane wave that is constant throughout the specimen volume. Therefore, to find
the induced dipole moment map, we can integrate over the back focal plane of the condenser
\begin{align}
  \bs{\mu}_{\text{ind}}(\mb{r}) \propto
  \intinf d\omega
  \int d\mb{r'}\ \eta(\omega)\bs{\mu}_{\text{em}}(\mb{r})
  \left[\bs{\mu}_{\text{abs}}(\mb{r})\cdot \mb{R}_{\mb{\hat{s}}}(\mb{r'})\mb{E}_{\text{bfp}}(\omega)\right] \label{eq:test}
\end{align}
where $\mb{r'}$ is the position in the back focal plane, $\mb{E}_{\text{bfp}}$
is the electric field direction in the back focal plane set by a linear
polarizer, and $\mb{R}_{\mb{\hat{s}}}(\mb{r'})$ is a rotation matrix that rotates the electric field in the back focal plane to the electric field in the specimen volume. 

Note that equation \ref{eq:test} depends on the paraxial approximation. Points
in the back focal plane that are far from the optical axis do not create perfect
plane waves in the specimen volume. We can explore more accurate models for high
NA illumination if necessary.

\section{Revised Model Summary}
On the detection side, the model remains the same. The intensity in the image plane
is given by
\begin{align}
  I_{\text{img}}(\mb{r''}) \propto \left|\mathcal{F}_{3D}\left\{
  \mb{\tilde{O}}_{\text{obj}}\tensor{\mathbf{G}}_{FF}\bs{\mu}_{\text{ind}}(\mb{r})
  \right\}
  \right|^2.
\end{align}

Under K\"{ohler} illumination the induced dipole moment map is
\begin{align}
  \bs{\mu}_{\text{ind}}(\mb{r}) \propto
  \intinf d\omega
  \int d\mb{r'}\ \eta(\omega)\bs{\mu}_{\text{em}}(\mb{r})
  \left[\bs{\mu}_{\text{abs}}(\mb{r})\cdot \mb{R}_{\mb{\hat{s}}}(\mb{r'})\mb{E}_{\text{bfp}}(\omega)\right] \label{eq:test2}
\end{align}
Under laser illumination the induced dipole moment is\begin{align}
  \bs{\mu}_{\text{ind}}(\mb{r}) \propto
  \bs{\mu}_{\text{em}}(\mb{r})
  \left[\bs{\mu}_{\text{abs}}(\mb{r})\cdot \mb{E}(\mb{r})\right] \label{eq:test3}
\end{align}
where $\mb{E}(\mb{r})$ is the electric field in the specimen volume. 

\section{Matrix Form of the Model}
Rudolph and Shalin suggested simplifying the model by defining a
\textit{polarization resolved intensity},
$\mb{I}_{\text{in}} = \left[I_{\text{in}}^x, I_{\text{in}}^y,
  I_{\text{in}}^z\right]$ that we could use to create a matrix form of the
model. Unfortunately, both the emission and absorption dipole maps appear in
equations \ref{eq:test2} and \ref{eq:test3}. This means that even if the
emission and absorption dipoles point in the same direction, the induced dipole
moments depend on the cross terms of the dipole direction. In other words, the
model is linear in
$\left[\mu_x^2, \mu_y^2, \mu_z^2, \mu_{x}\mu_y, \mu_x\mu_z, \mu_y\mu_z\right]$,
not just $\left[\mu_x^2, \mu_y^2, \mu_z^2\right]$. If we define a polarization resolved intensity vector it will need six terms, not three.

Backer and Moerner made a similar simplification on the detection side to speed
up their deconvolutions \cite{backer}.

\bibliography{report}{}
\bibliographystyle{unsrt}

\end{document}

