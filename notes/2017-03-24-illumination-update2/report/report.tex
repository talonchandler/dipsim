\documentclass[11pt]{article}

%%%%%%%%%%%%
% Packages %
%%%%%%%%%%%%
\hyphenpenalty=10000
\usepackage{tocloft}
\renewcommand\cftsecleader{\cftdotfill{\cftdotsep}}
\def\undertilde#1{\mathord{\vtop{\ialign{##\crcr
$\hfil\displaystyle{#1}\hfil$\crcr\noalign{\kern1.5pt\nointerlineskip}
$\hfil\tilde{}\hfil$\crcr\noalign{\kern1.5pt}}}}}
\usepackage{xcolor}
\usepackage{hyperref}
\usepackage{epstopdf}
\usepackage{braket}
\usepackage{upgreek}
\usepackage{caption}
\usepackage{booktabs}
\usepackage{subcaption}
\usepackage{amssymb,latexsym,amsmath,gensymb}
\usepackage{latexsym}
\usepackage{graphicx}
\usepackage{float}
\usepackage{enumitem}
\usepackage{pdflscape}
\usepackage{url}
\usepackage{tikz, calc}
\usetikzlibrary{shapes.geometric, arrows, calc}
\tikzstyle{norm} = [rectangle, rounded corners, minimum width=2cm, minimum height=1cm,text centered, draw=black]
\tikzstyle{arrow} = [thick, ->, >=stealth]

\providecommand{\e}[1]{\ensuremath{\times 10^{#1}}} 
\providecommand{\mb}[1]{\mathbf{#1}}
\providecommand{\mh}[1]{\mathbf{\hat{#1}}}
\providecommand{\bs}[1]{\boldsymbol{#1}} 
\providecommand{\intinf}{\int_{-\infty}^{\infty}}
\providecommand{\fig}[4]{
  % filename, width, caption, label
\begin{figure}[h]
 \captionsetup{width=1.0\linewidth}
 \centering
 \includegraphics[width = #2\textwidth]{#1}
 \caption{#3}
 \label{fig:#4}
\end{figure}
}

\newcommand{\tensor}[1]{\overset{\text{\tiny$\leftrightarrow$}}{\mb{#1}}}
\newcommand{\tunderbrace}[2]{\underbrace{#1}_{\textstyle#2}}
\providecommand{\figs}[7]{
  % filename1, filename2, caption1, caption2, label1, label2, shift
\begin{figure}[H]
\centering
\begin{minipage}[b]{.45\textwidth}
  \centering
  \includegraphics[width=1.0\linewidth]{#1}
  \captionsetup{justification=justified, singlelinecheck=true}
  \caption{#3}
  \label{fig:#5}
\end{minipage}
\hspace{2em}
\begin{minipage}[b]{.45\textwidth}
  \centering
  \includegraphics[width=1.0\linewidth]{#2}
  \vspace{#7em}
  \captionsetup{justification=justified}
  \caption{#4}
  \label{fig:#6}
\end{minipage}
\end{figure}
}
\makeatletter

\providecommand{\code}[1]{
\begin{center}
\lstinputlisting{#1}
\end{center}
}

%%%%%%%%%%%
% Spacing %
%%%%%%%%%%%
% Margins
\usepackage[
top    = 1.5cm,
bottom = 1.5cm,
left   = 1.5cm,
right  = 1.5cm]{geometry}

% Indents, paragraph space
\usepackage{parskip} 

% Section spacing
\usepackage{titlesec}
\titlespacing*{\title}
{0pt}{0ex}{0ex}
\titlespacing*{\section}
{0pt}{0ex}{0ex}
\titlespacing*{\subsection}
{0pt}{0ex}{0ex}
\titlespacing*{\subsubsection}
{0pt}{0ex}{0ex}

% Line spacing
\linespread{1.1}

%%%%%%%%%%%%
% Document %
%%%%%%%%%%%%
\begin{document}
\title{\vspace{-2.5em}Dipole Illumination Model Correction\vspace{-1em}}
\author{Talon Chandler}% and Patrick La Rivi\`ere}
\date{\vspace{-1em}March 24, 2017\vspace{-1em}}
\maketitle

\section{Summary}
Our current model for the induced dipole moment created by illuminating a single
fluorophore with a monochromatic plane wave is
\begin{align}
  \bs{\mu}_{\text{ind}} \propto \hat{\bs{\mu}}_{\text{em}} \left[\hat{\bs{\mu}}_{\text{abs}}\cdot \mb{E}\right].
\end{align}

In these notes I will re-derive this model and correct several errors. The corrected
model is
\begin{align}
  \bs{\mu}_{\text{ind}} \propto \hat{\bs{\mu}}_{\text{em}} \left|\hat{\bs{\mu}}_{\text{abs}}^{\dagger}\mb{A}\right|
\end{align}
where $\mb{A}$ is the complex envelope (also known as the generalized Jones
vector).

\section{Math Preliminaries}
\subsection{Set Definitions}
\begin{align*}
  \mathbb{R}:&\ \text{the set of real numbers}\\
  \mathbb{R}^n:&\ \text{the set of $n$ dimensional vectors with real components}\\
  \mathbb{C}:&\ \text{the set of complex numbers}\\
  \mathbb{C}^n:&\ \text{the set of $n$ dimensional vectors with complex components}\\
  \mathbb{S}^n &= \left\{x\in \mathbb{R}^{n}: |x| = 1 \right\}:\ \text{the set of real unit vectors in $n$ dimensions.}\\
  \mathbb{T}^n &= \left\{x\in \mathbb{C}^{n}: |x| = 1 \right\}:\ \text{the set of complex unit vectors in $n$ dimensions.}\\
  [a, b) &= \left\{x\in \mathbb{R}: a\leq x < b\right\}:\ \text{shorthand for subsets of the real numbers. }
\end{align*}
\subsection{Useful Operations With Complex Vectors}
\label{math}
Consider $\mb{x}, \mb{y} \in \mathbb{C}^n$.
\begin{enumerate}
\item The conjugate transpose of $\mb{x}$ is denoted by a dagger ($\mb{x}^{\dagger}$).
\item $|\mb{x}|^2 = \mb{x}^{\dagger}\mb{x}$.
\item $\mb{x}^{\dagger}\mb{y} = \mb{x}\cdot \mb{y}$ if $\mb{x}, \mb{y}\in \mathbb{R}^n$
\item $(\mb{x}\mb{y}^{\dagger})^{\dagger} = \mb{y}\mb{x}^{\dagger}$
\end{enumerate}

\section{Representations of Polarized Plane Waves}
The time dependent electric field $\mathcal{E}$ of a polarized monochromatic
plane wave is given by
\begin{equation}
  \mathcal{E}(\mb{r},t) = E_{o,x}\cos(k_{x}r_{x} - \omega t)\mh{x}\ + \\
                     E_{o,y}\cos(k_{y}r_{y} - \omega t - \varphi_y)\mh{y}\ +\\
                     E_{o,z}\cos(k_{z}r_{z} - \omega t - \varphi_z)\mh{z}
                       \label{eq:plane}
\end{equation}
where $\mb{E}_{o} = [E_{o,x}, E_{o,y}, E_{o,z}]^T \in \mathbb{R}^3$ is the real
electric field amplitude, $\mb{k} = [k_x, k_y, k_z]^T \in \mathbb{R}^3$ is the
wave vector which points in the direction of propagation,
$\mb{r} = [r_x, r_y, r_z]^T \in \mathbb{R}^3$ is the position,
$\omega \in \mathbb{R}$ is the angular frequency, and $\varphi_y$ ($\varphi_z$)
$\in [0, 2\pi)$ is the phase shift of the y (z) component. For equation
\ref{eq:plane} to represent a physically realizable polarized plane wave, we
require that $\mb{E}\cdot \mb{k} = 0$ so that the plane wave is transverse, and
$|\mb{k}| = \frac{\omega}{c}$ so that the wave can propagate. 

The complex spatial electric field $\mb{E}$ for a monochromatic polarized plane wave is given by
\begin{equation}
  \mb{E}(\mb{r}) = \mb{A}e^{j\mb{k}\cdot\mb{r}}
  \label{eq:factor}
\end{equation}
where $\mb{E} \in \mathbb{C}^3$, and $\mb{A} \in \mathbb{C}^3$ is the complex
envelope. We can recover the time dependent electric field from the complex spatial
electric field using
\begin{equation}
  \mathcal{E}(\mb{r},t) = \text{Re}\left\{\mb{E}(\mb{r})e^{-j\omega t}\right\}
\end{equation}

We can view equation \ref{eq:factor} as a factorization of $\mb{E}$ into a
spatially dependent part ($e^{j\mb{k}\cdot\mb{r}}$) and a spatially independent
part ($\mb{A}$). $\mb{A}$ is called the complex envelope \cite{saleh} or the
generalized Jones vector \cite{azzam}. The conventional Jones vector $\mb{J}$
only considers polarized plane waves traveling along the $z$ axis, so
$\mb{J}\in\mathbb{C}^2$ suffices to describe the polarization state. $\mb{A}$
describes the polarization state for plane waves traveling in any direction, so
$\mb{A} \in \mathbb{C}^3$ is required.

\section{Power Absorbed By A Dipole}
The power absorbed by a single dipole with absorption dipole moment
$\bs{\mu}_{\text{abs}}$ illuminated by a polarized plane wave is \cite{nov}
\begin{equation}
  P_{\text{abs}} \propto |\hat{\bs{\mu}}_{\text{abs}}^{\dagger} \mb{E}|^2
  \label{eq:power}
\end{equation}
where $\hat{\bs{\mu}}_{\text{abs}} \in \mathbb{T}^3$ is the absorption dipole
moment. Notice that we've allowed $\hat{\bs{\mu}}_{\text{abs}}$ to be a unit
complex vector instead of restricting it to be a unit real vector. Although
we'll focus on the case of $\hat{\bs{\mu}}_{\text{abs}} \in \mathbb{S}^3$
experimentally, allowing $\hat{\bs{\mu}}_{\text{abs}} \in \mathbb{T}^3$
simplifies the notation and allows us to consider complex transition moments
(transitions that are excited by circularly polarized light) if required. 

\subsection{Absorbed Power Is Independent of Phase}
Equation \ref{eq:power} extends to generalized Jones vectors. Plugging
\ref{eq:factor} into \ref{eq:power} and using the operations in section \ref{math} gives
\begin{align}
  P_{\text{abs}} &\propto |\hat{\bs{\mu}}_{\text{abs}}^{\dagger} \mb{A}e^{j\mb{k}\cdot\mb{r}}|^2\\
  P_{\text{abs}} &\propto (\hat{\bs{\mu}}_{\text{abs}}^{\dagger} \mb{A}e^{j\mb{k}\cdot\mb{r}})^{\dagger}(\hat{\bs{\mu}}_{\text{abs}}^{\dagger} \mb{A}e^{j\mb{k}\cdot\mb{r}})\\
  P_{\text{abs}} &\propto (\hat{\bs{\mu}}_{\text{abs}}^{\dagger} \mb{A})^{\dagger}(\hat{\bs{\mu}}_{\text{abs}}^{\dagger} \mb{A})e^{j\mb{k}\cdot\mb{r}}e^{-j\mb{k}\cdot\mb{r}}\\
  P_{\text{abs}} &\propto |\hat{\bs{\mu}}_{\text{abs}}^{\dagger} \mb{A}|^2
  \label{eq:power2}                   
\end{align}
This confirms Rudolph and Shalin's claim that absorbed power is independent of
phase. 

\subsection{Sum Over Source}
Finally, we show that if a dipole is illuminated by two plane waves with
independent phase (say by two lasers that are not phase locked, or by
illuminating two points on the back focal plane of a condenser with a thermal
source) then the total absorbed power is the sum of the absorbed powers
due to the individual plane waves. Let $\mb{E}_1$ and
$\mb{E}_2\text{exp}\{j\undertilde{\phi}\}$ be the complex spatial electric
fields created by the first and second illuminating plane waves where
$\undertilde{\phi} \sim U(0, 2\pi)$ is a uniformly distributed phase difference
between the two plane waves. The expected value of the absorbed power is
\begin{align}
  E[\undertilde{P}{\tiny{\text{abs}}}] &\propto E\left[|\hat{\bs{\mu}}_{\text{abs}}^{\dagger} (\mb{E}_1 + \mb{E}_2\text{exp}\{j\undertilde{\phi}\})|^2\right]\\
  E[\undertilde{P}{\tiny{\text{abs}}}] &\propto E\left[\left[\hat{\bs{\mu}}_{\text{abs}}^{\dagger} (\mb{E}_1 + \mb{E}_2\text{exp}\{j\undertilde{\phi}\})\right]^{\dagger}\left[\hat{\bs{\mu}}_{\text{abs}}^{\dagger} (\mb{E}_1 + \mb{E}_2\text{exp}\{j\undertilde{\phi}\})\right]\right]\\
  E[\undertilde{P}{\tiny{\text{abs}}}] &\propto E\left[\left[\hat{\bs{\mu}}_{\text{abs}}^{\dagger} (\mb{E}_1 + \mb{E}_2\text{exp}\{j\undertilde{\phi}\})(\mb{E}_1^{\dagger} + \mb{E}_2^{\dagger}\text{exp}\{-j\undertilde{\phi}\})\hat{\bs{\mu}}_{\text{abs}}\right]^{\dagger}\right]\\
  E[\undertilde{P}{\tiny{\text{abs}}}] &\propto E\left[\left[\hat{\bs{\mu}}_{\text{abs}}^{\dagger} (\mb{E}_1\mb{E}_1^{\dagger} + \mb{E}_2\mb{E}_1^{\dagger}\text{exp}\{j\undertilde{\phi}\} + \mb{E}_1^{\dagger}\mb{E}_2\text{exp}\{-j\undertilde{\phi}\} + \mb{E}_2\mb{E}_2^{\dagger})\hat{\bs{\mu}}_{\text{abs}}\right]^{\dagger}\right]
\intertext{Pulling the expectation inside and using $E[\text{exp}\{j\undertilde{\phi}\}] = 0$ gives}
  E[\undertilde{P}{\tiny{\text{abs}}}] &\propto \left[\hat{\bs{\mu}}_{\text{abs}}^{\dagger} (\mb{E}_1\mb{E}_1^{\dagger} + \mb{E}_2\mb{E}_2^{\dagger})\hat{\bs{\mu}}_{\text{abs}}\right]^{\dagger}\\
  E[\undertilde{P}{\tiny{\text{abs}}}] &\propto \left[\hat{\bs{\mu}}_{\text{abs}}^{\dagger}\mb{E}_1\mb{E}_1^{\dagger}\hat{\bs{\mu}}_{\text{abs}} + \hat{\bs{\mu}}_{\text{abs}}^{\dagger}\mb{E}_2\mb{E}_2^{\dagger}\hat{\bs{\mu}}_{\text{abs}}\right]^{\dagger}\\
  E[\undertilde{P}{\tiny{\text{abs}}}] &\propto |\hat{\bs{\mu}}_{\text{abs}}^{\dagger}\mb{E}_1|^2 + |\hat{\bs{\mu}}_{\text{abs}}^{\dagger}\mb{E}_2|^2\\
  E[\undertilde{P}{\tiny{\text{abs}}}] &\propto P_{\text{abs},1} + P_{\text{abs},2}
\end{align}
This confirms Rudolph and Shalin's claim that independent plane waves excite a
fluorophore independently. We can extend the argument to an arbitrary number of
independent plane waves which allows us to integrate over all plane waves
incident on the fluorophore to find the total absorbed power.

\section{Dipole Emission Pattern}
The emitted electric field pattern is proportional to the Green's
tensor multiplied by the emission dipole moment of the fluorophore
\begin{align}
  \mb{E}_{\text{em}}(\mb{r}) \propto \mb{G}\hat{\bs{\mu}}_{\text{em}}.\label{eq:eight}
\end{align}
where $\hat{\bs{\mu}}_{\text{em}} \in \mathbb{T}^3$.
The power emitted by a fluorophore is proportional to the power absorbed
\begin{align}
  P_{\text{em}} &\propto P_{\text{abs}}. 
\end{align}
Using equation $P_{\text{em}} \propto |\mb{E}_{\text{em}}(\mb{r})|^2$ and \ref{eq:power2} gives
\begin{align}
  |\mb{E}_{\text{em}}(\mb{r})|^2 &\propto |\hat{\bs{\mu}}_{\text{abs}}^{\dagger}\mb{A}|^2.\label{eq:twenty}
\end{align}
Taking the square root of \ref{eq:twenty} and combining with \ref{eq:eight} gives the final result
\begin{align}
  \mb{E}_{\text{em}}(\mb{r}) \propto \mb{G}\hat{\bs{\mu}}_{\text{em}}|\hat{\bs{\mu}}^{\dagger}_{\text{abs}}\mb{A}|. \label{eq:final}
\end{align}

\section{Induced Dipole Moment}
We can split equation \ref{eq:final} into two parts using the induced dipole
moment
\begin{align}
  \bs{\mu}_{\text{ind}} \equiv \hat{\bs{\mu}}_{\text{em}}|\hat{\bs{\mu}}^{\dagger}_{\text{abs}}\mb{A}|\\
  \mb{E}_{\text{em}}(\mb{r}) \propto \mb{G}\bs{\mu}_{\text{ind}}
\end{align}
where $\bs{\mu}_{\text{ind}} \in \mathbb{C}^3$. 

\section{K\"{o}hler Illumination}
The work in these notes extends to the previous note set. The induced dipole
moment created by K\"{o}hler illumination is given by
\begin{align}
  \bs{\mu}_{\text{ind}} \propto
  \bs{\mu}_{\text{em}} \int d\mb{r'}
  \left|\bs{\mu}_{\text{abs}}^{\dagger} \mb{R}(\mb{r'})\mb{A}_{\text{bfp}}\right| \label{eq:test2}
\end{align}
Shalin suggested that we separate the model into three parts by moving
$\bs{\mu}_{\text{em}}$ and $\bs{\mu}_{\text{abs}}$ outside of the integral. This
would allow us to calculate the ``effective electric field'' or ``polarization resolved
electric field''. The addition of the absolute value bars prevent us from making this simplification.

With equation \ref{eq:test2} in view, we can see why the absolute value bars are
required. Without the absolute value bars,
$\bs{\mu}_{\text{abs}}^{\dagger}(\mb{r})
\mb{R}_{\mb{\hat{s}}}(\mb{r'})\mb{A}_{\text{bfp}} \in [-1, 1]$, so the integral
can evaluate to zero. With the absolute value bars,
$|\bs{\mu}_{\text{abs}}^{\dagger}(\mb{r})
\mb{R}_{\mb{\hat{s}}}(\mb{r'})\mb{A}_{\text{bfp}}| \in [0, 1]$ and each point in
the back focal plane contributes to the induced dipole moment.

\section{Total Power Emitted By A Dipole}
\figs{../figures/scene1}{../figures/intensity1}{Scene schematic. A single dipole (see arrow in the
  front focal plane) under $x$-polarized K\"{o}hler illumination (see arrow in
  the back focal plane).\\}{Total power emitted by a single dipole
  ($|\bs{\mu}_{\text{ind}}|^2$) as a function of direction under the illumination
  geometry in Figure \ref{fig:scene}. Dots indicate where the Cartesian axes
  intersect the sphere. Arbitrary units.}{scene}{int}{-0.2}

Figure \ref{fig:scene} shows a simple K\"{o}hler illumination geometry where
$x$-polarized light illuminates a dipole with the optical axis along $z$. Figure
\ref{fig:int} shows the total power emitted by the dipole as a function of its
direction. In practice, we will only collect a fraction of the total power
emitted, but Figure \ref{fig:int} shows the total power---as if we had
detectors on all sides of the dipole. We can see that dipoles
aligned along the $x$ axis will emit (and absorb) the most power, while dipoles
aligned along the $y$ or $z$ axes will emit very little power.

Figure \ref{fig:illum} shows the total power emitted by a single dipole under
different back aperture radii, $\rho$, with constant illumination power. Our
model assumes that the lens is large enough to map each point in the back focal
plane to a plane wave in the front focal plane, so the only parameter of
interest is $\frac{\rho}{f}$, the ratio of the back aperture radius to the focal
length of the lens.

Figure \ref{fig:illum} also shows how the results change when we neglect (second
row) and keep (third row) the absolute value bars inside the integral in
equation \ref{eq:test2}. When we neglect the absolute value bars, the model
predicts that as we open the back aperture the emitted power pattern is not
mirror symmetric about the $x-y$ plane. When we keep the absolute value bars,
the absorbed power pattern is symmetric about the $x-y$ plane. Also, notice that
$+z$ and $-z$ aligned dipoles emit more power as we open the back aperture (you
may have to zoom in on the bottom row to see the difference---the $+z$ axis in
the bottom right plot is $\sim 300$ compared to $\sim 0$ in the bottom left
plot). 

\fig{../figures/compare_illum}{1.0}{Top row---schematics of the illumination
  geometries. Middle row---\textbf{incorrect} emitted power pattern if the absolute value bars are neglected. Bottom row---\textbf{correct} emitted power pattern if the absolute value bars are retained. Columns---the back aperture radius increases from left to right.}{illum}

\section{Aside: Generalized Jones Vectors}
The generalized Jones vector in equation \ref{eq:factor} is going to be extremely
useful for analyzing light fields traveling through crystals. Ortega
\cite{ortega} and Azzam \cite{azzam} have started developing the generalized
Jones calculus, but they haven't applied the results to stacks of
crystals. Foreman and Torok \cite{foreman} have used generalized Jones vectors
to analyze high-NA imaging systems, but they haven't applied their results to
transmission microscopes like the LC-Polscope. More progress on this front soon!

\bibliography{report}{}
\bibliographystyle{unsrt}

\end{document}

