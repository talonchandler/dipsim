\documentclass[11pt]{article}

%%%%%%%%%%%%
% Packages %
%%%%%%%%%%%%
\hyphenpenalty=10000
\usepackage{tocloft}
\renewcommand\cftsecleader{\cftdotfill{\cftdotsep}}
\def\undertilde#1{\mathord{\vtop{\ialign{##\crcr
$\hfil\displaystyle{#1}\hfil$\crcr\noalign{\kern1.5pt\nointerlineskip}
$\hfil\tilde{}\hfil$\crcr\noalign{\kern1.5pt}}}}}
\usepackage{xcolor}
\usepackage{hyperref}
\usepackage{epstopdf}
\usepackage{braket}
\usepackage{upgreek}
\usepackage{caption}
\usepackage{booktabs}
\usepackage{subcaption}
\usepackage{amssymb,latexsym,amsmath,gensymb}
\usepackage{latexsym}
\usepackage{graphicx}
\usepackage{float}
\usepackage{enumitem}
\usepackage{pdflscape}
\usepackage{url}
\usepackage{tikz, calc}
\usetikzlibrary{shapes.geometric, arrows, calc}
\tikzstyle{norm} = [rectangle, rounded corners, minimum width=2cm, minimum height=1cm,text centered, draw=black]
\tikzstyle{arrow} = [thick, ->, >=stealth]

\providecommand{\e}[1]{\ensuremath{\times 10^{#1}}} 
\providecommand{\mb}[1]{\mathbf{#1}}
\providecommand{\mh}[1]{\mathbf{\hat{#1}}}
\providecommand{\bs}[1]{\boldsymbol{#1}} 
\providecommand{\intinf}{\int_{-\infty}^{\infty}}
\providecommand{\fig}[4]{
  % filename, width, caption, label
\begin{figure}[h]
 \captionsetup{width=1.0\linewidth}
 \centering
 \includegraphics[width = #2\textwidth]{#1}
 \caption{#3}
 \label{fig:#4}
\end{figure}
}

\newcommand{\tensor}[1]{\overset{\text{\tiny$\leftrightarrow$}}{\mb{#1}}}
\newcommand{\tunderbrace}[2]{\underbrace{#1}_{\textstyle#2}}
\providecommand{\figs}[7]{
  % filename1, filename2, caption1, caption2, label1, label2, shift
\begin{figure}[H]
\centering
\begin{minipage}[b]{.45\textwidth}
  \centering
  \includegraphics[width=1.0\linewidth]{#1}
  \captionsetup{justification=justified, singlelinecheck=true}
  \caption{#3}
  \label{fig:#5}
\end{minipage}
\hspace{2em}
\begin{minipage}[b]{.45\textwidth}
  \centering
  \includegraphics[width=1.0\linewidth]{#2}
  \vspace{#7em}
  \captionsetup{justification=justified}
  \caption{#4}
  \label{fig:#6}
\end{minipage}
\end{figure}
}
\makeatletter

\providecommand{\code}[1]{
\begin{center}
\lstinputlisting{#1}
\end{center}
}

%%%%%%%%%%%
% Spacing %
%%%%%%%%%%%
% Margins
\usepackage[
top    = 1.5cm,
bottom = 1.5cm,
left   = 1.5cm,
right  = 1.5cm]{geometry}

% Indents, paragraph space
\usepackage{parskip} 

% Section spacing
\usepackage{titlesec}
\titlespacing*{\title}
{0pt}{0ex}{0ex}
\titlespacing*{\section}
{0pt}{0ex}{0ex}
\titlespacing*{\subsection}
{0pt}{0ex}{0ex}
\titlespacing*{\subsubsection}
{0pt}{0ex}{0ex}

% Line spacing
\linespread{1.1}

%%%%%%%%%%%%
% Document %
%%%%%%%%%%%%
\begin{document}
\title{\vspace{-2.5em} Inconsistency In ``Rapid determination of the three-dimensional orientation of single molecules''\vspace{-1em}}
\author{Talon Chandler}% and Patrick La Rivi\`ere}
\date{\vspace{-1em}April 25, 2017\vspace{-1em}}
\maketitle

\section{Introduction}
There is an inconsistency in John T. Fourkas' paper titled ``Rapid determination
of the three dimensional orientation of single molecules''\cite{fourkas}. In
section 2 I describe the inconsistency. In section 3 I follow Fourkas'
calculation in detail and correct it. In section 4 I show the results of the
corrected calculation and show that the new results are consistent. Finally, in
section 5 I discuss how these notes will affect other papers and the
applicability of these results.

\section{Inconsistency}
Equations 4 and 5 of \cite{fourkas} model the fraction of the total intensity
from a single fluorophore collected by an objective with a polarizer in the back
focal plane as a function of polarizer orientation, dipole orientation
($\Theta$, $\Phi$), and half angle of the light collection cone
($\alpha = \sin^{-1}(\text{NA}/n)$). I expect that as $\alpha$ approaches
$\frac{\pi}{2}$ (or a NA approaches $n$) the fraction of the total power
radiated by the dipole that is collected by the lens should approach
$\frac{1}{2}$
\begin{align}
  \lim_{\alpha\rightarrow \frac{\pi}{2}} \frac{I_{0}(\Theta, \Phi, \alpha) + I_{90}(\Theta, \Phi, \alpha)}{I_{\text{tot}}(\Theta, \Phi)} = \frac{1}{2}. \label{eq:prop}
\end{align}

First, I reproduce equations 4 and 5 from \cite{fourkas}
\begin{subequations}
\begin{align}
    I_{0}(\Theta, \Phi, \alpha) &= I_{\text{tot}}(t, t+\tau)(A + B\sin^{2}{\Theta} + C\sin^{2}{\Theta} \cos{2 \Phi})\\
    I_{45}(\Theta, \Phi, \alpha) &= I_{\text{tot}}(t, t+\tau)(A + B\sin^{2}{\Theta} + C\sin^{2}{\Theta}\sin{2 \Phi})\\
    I_{90}(\Theta, \Phi, \alpha) &= I_{\text{tot}}(t, t+\tau)(A + B\sin^{2}{\Theta} - C\sin^{2}{\Theta} \cos{2 \Phi})\\
  I_{135}(\Theta, \Phi, \alpha) &= I_{\text{tot}}(t, t+\tau)(A + B\sin^{2}{\Theta} - C\sin^{2}{\Theta} \sin{2 \Phi}),
\end{align}\label{eq:int1}
\end{subequations}
and
\vspace{-1em}
\begin{subequations}
\begin{align}
  A &= \frac{1}{6} - \frac{1}{4}\cos\alpha + \frac{1}{12}\cos^3\alpha\\
  B &= \frac{1}{8}\cos\alpha - \frac{1}{8}\cos^3\alpha\\
  C &= \frac{7}{48} - \frac{1}{16} \cos{\alpha } - \frac{1}{16} \cos^{2}{\alpha } - \frac{1}{48} \cos^{3}{\alpha }.
\end{align}\label{eq:coeff1}
\end{subequations}
Next, I take the limit proposed in equation \ref{eq:prop}
\begin{align}
  \lim_{\alpha\rightarrow \frac{\pi}{2}} \frac{I_{0}(\Theta, \Phi, \alpha) + I_{90}(\Theta, \Phi, \alpha)}{I_{\text{tot}}(\Theta, \Phi)} &= \lim_{\alpha\rightarrow \frac{\pi}{2}} 2\left(A(\alpha) + B(\alpha)\sin^2\Theta\right) =2\left(\frac{1}{6} + 0\right) = \frac{1}{3} \neq \frac{1}{2}.
\end{align}
Fourkas' model predicts that only $\frac{1}{3}$ of the total power radiated by the
fluorophore is collected by an objective with a collection half angle of $\frac{\pi}{2}$.

\section{Corrected Calculation}
I start by following Fourkas and define the $z$ axis as the optical axis of the
objective. I express a directional unit vector $\hat{\mb{r}}$ and the
fluorescence emission dipole $\hat{\bs{\mu}}_{\text{em}}$ in spherical
coordinates as follows
\begin{align}
  \hat{\mb{r}} &= \sin\theta\cos\phi\hat{\mb{i}} + \sin\theta\sin\phi\hat{\mb{j}} + \cos\theta\hat{\mb{k}}\label{eq:r_coords}\\
  \hat{\bs{\mu}}_{\text{em}} &= \sin\Theta\cos\Phi\hat{\mb{i}} + \sin\Theta\sin\Phi\hat{\mb{j}} + \cos\Theta\hat{\mb{k}}.\label{eq:mu_coords}
\end{align}
Next, I calculate the generalized Jones vector $\mb{A}$ (GJV or the complex
envelope) along a direction $\hat{\mb{r}}$ in the far field
\begin{align}
\mb{\hat{A}}_{\text{ff}}(\hat{\mb{r}}) &= \mh{r}\times\hat{\bs{\mu}}_{\text{em}}\times\mh{r} \label{eq:fourkasa}\\ &= (\tilde{\mb{I}} - \hat{\mb{r}}\hat{\mb{r}}^{\dagger})\hat{\bs{\mu}}_{\text{em}} \label{eq:greena} 
\end{align}
where $\mb{I}$ is the identity matrix and ${}^{\dagger}$ is the adjoint
operator. Notice that $(\tilde{\mb{I}} - \hat{\mb{r}}\hat{\mb{r}}^{\dagger})$ is
the Green's tensor with the phase excluded. Also notice that $\mb{\hat{A}}_{\text{ff}}(\mh{r})$ is
not the electric field---it is the GJV because I have stopped keeping track of
the phase. I will follow Fourkas and use equation \ref{eq:fourkasa}, but I have
written equation \ref{eq:greena} to relate this work to future work with the
Green's tensor.

Next, I model the action of an ideal, infinity corrected, polarization
preserving objective on the GJV. The lens rotates the GJV so that it is
perpendicular to the $z$ axis at every point on the unit sphere---we can model this by rotating the ray by $-\phi$ about the $z$ axis, by $-\theta$ about the $y$ axis, then by $\phi$ about $z$ axis. In matrix form the position-dependent rotation matrix is
\begin{align}  
  \tilde{\mb{R}}(\mh{r}) = \begin{bmatrix} \cos\theta\cos^2\phi + \sin^2\phi & (\cos\theta -1)\sin\phi\cos\phi & -\sin\theta\cos\phi\\ (\cos\theta - 1)\sin\phi\cos\phi & \cos\theta\sin^2\phi + \cos^2\phi & -\sin\theta\sin\phi \\ \sin\theta\cos\phi& \sin\theta\sin\phi & \cos\theta \end{bmatrix},\label{eq:matrix}
\end{align}
which is identical to the equation 2 in \cite{fourkas}.

Next, I express the transmission axis of the polarizer $\mh{P}$ using
\begin{align}
  \hat{\mb{P}} = \cos\phi_{\text{pol}}\hat{\mb{i}} + \sin\phi_{\text{pol}}\hat{\mb{j}}. \label{eq:polarizer}
\end{align}
I model the action of the polarizer in the back focal plane by taking the
dot product of the generalized Jones vector with the polarizer axis $\mh{P}$ and multiplying by the polarizer axis direction. The GJV in the back focal plane of the objective is given by
\begin{align}
  \mb{A}_{\text{bfp}}(\mh{r}) = \mh{P}\left[\mh{P}\cdot\tilde{\mb{R}}(\mh{r})(\mh{r}\times\hat{\bs{\mu}}_{\text{em}}\times\mh{r})\right].\label{eq:approx}
\end{align}
Notice that I am following Fourkas and mapping the GJV directly to the back focal plane
without considering the change of coordinates. I will discuss this approximation in section \ref{discussion}.

Next, I model the detection process by squaring the GJV and integrating over
the cap of the sphere collected by the objective, $\Omega$. The complete model
of the detected intensity is
\begin{align}
  I_{\phi_{\text{pol}}}(\Theta, \Phi) \propto \int_{\Omega}d\mh{r}\ \left|\mh{P}\left[\mh{P}\cdot\tilde{\mb{R}}(\mh{r})(\mh{r}\times\hat{\bs{\mu}}_{\text{em}}\times\mh{r})\right]\right|^2. \label{eq:model1}
\end{align}
Next, I calculate the total power emitted by a unit dipole
\begin{align}
  I_{\text{tot}} = \int_{\mathbb{S}^2}d\mh{r}\ \left|\mh{r}\times\hat{\bs{\mu}}_{\text{em}}\times\mh{r}\right|^2 = \int_0^{2\pi}d\phi\int_0^{\pi}d\theta\sin^3\theta = \frac{8\pi}{3}
\end{align}
Finally, I calculate the fraction of the total power detected
\begin{align}
  \frac{I_{\phi_{\text{pol}}}(\Theta, \Phi)}{I_{\text{tot}}} = \frac{3}{8\pi}\int_{\Omega}d\mh{r}\ \left|\mh{P}\left[\mh{P}\cdot\tilde{\mb{R}}(\mh{r})(\mh{r}\times\hat{\bs{\mu}}_{\text{em}}\times\mh{r})\right]\right|^2\label{eq:final}
\end{align}
where the total radiated power $I_{\text{tot}}$ was found using equation \ref{eq:model1} with no polarizer or objective and by integrating over all directions. 

Notice that I normalized using the total radiated intensity. Fourkas incorrectly normalized the GJV before calculating the intensity as follows 
\begin{align}
  \frac{I_{\phi_{\text{pol}}}(\Theta, \Phi)}{I_{\text{tot}}} &\neq \int_{\Omega}d\mh{r}\left|\left(\frac{\mb{A}_{\phi_{\text{pol}}}}{\mb{A}_{\text{tot}}}\right)\right|^2\label{eq:incorrect}\\
  \intertext{It's not physically clear what $\mb{A}_{\text{tot}}$ represents, but I can make a good guess about how Fourkas calculated it}
  \mb{A}_{\text{tot}} &= \int_{\mathbb{S}^2} d\mh{r} (\mh{r}\times\hat{\bs{\mu}}_{\text{em}}\times\mh{r}) = \int_0^{2\pi}d\phi\int_0^{\pi}d\theta\sin\theta = 4\pi\label{eq:atot}\\
  \intertext{Plugging \ref{eq:atot} into \ref{eq:incorrect} gives}
  \frac{I_{\phi_{\text{pol}}}(\Theta, \Phi)}{I_{\text{tot}}} &\neq \frac{1}{16\pi}\int_{\Omega}d\mh{r}\ \left|\mh{P}\left[\mh{P}\cdot\tilde{\mb{R}}(\mh{r})(\mh{r}\times\hat{\bs{\mu}}_{\text{em}}\times\mh{r})\right]\right|^2\label{eq:four_final}
\end{align}
Notice that equation \ref{eq:four_final} and equation \ref{eq:final} differ by a factor of $\frac{3}{2}$.

\section{Results}
I evaluated equations \ref{eq:final} and \ref{eq:four_final} by substituting equations \ref{eq:r_coords}, \ref{eq:mu_coords}, \ref{eq:matrix}, and \ref{eq:polarizer} and simplifying using SymPy, an open-source computer algebra system. Equation \ref{eq:four_final} yields Fourkas' published expressions. Equation \ref{eq:final} yields Fourkas' expressions multiplied by a factor of $\frac{3}{2}$. We can make this correction by changing the
coefficients of $A$, $B$, and $C$ to 
\begin{subequations}
\begin{align}
  A_{\text{new}} &= \frac{1}{4} - \frac{3}{8} \cos{\alpha } + \frac{1}{8} \cos^{3}{\alpha }\\
  B_{\text{new}} &= \frac{3}{16} \cos{\alpha } - \frac{3}{16} \cos^{3}{\alpha }\\
  C_{\text{new}} &= \frac{7}{32} - \frac{3}{32} \cos{\alpha } - \frac{3}{32} \cos^{2}{\alpha } - \frac{1}{32} \cos^{3}{\alpha}.
\end{align}\label{eq:coeff2}
\end{subequations}

Fourkas' published inversion expressions still hold if we use $A_{\text{new}}$,
$B_{\text{new}}$, and $C_{\text{new}}$.

Finally, I confirm that taking the limit with the new expressions approaches $\frac{1}{2}$
\begin{align}
  \lim_{\alpha\rightarrow \frac{\pi}{2}} \frac{I_{0}(\Theta, \Phi) + I_{90}(\Theta, \Phi)}{I_{\text{tot}}(\Theta, \Phi)} &= \lim_{\alpha\rightarrow \frac{\pi}{2}} 2\left(A_{\text{new}} + B_{\text{new}}\sin^2\Theta\right) =2\left(\frac{1}{4} + 0\right) = \frac{1}{2}  
\end{align}

\section{Discussion} \label{discussion} Fourkas' expressions underpredict the
fraction of the total power collected by an objective with a polarizer by a
constant fraction of $\frac{3}{2}$. Therefore, using Fourkas' expressions to
find $I_{\text{tot}}$ will result in overpredictions of $I_{\text{tot}}$, but
the incorrect expressions will not cause any error in the predictions of molecular
orientation.

Lu and Bout \cite{bout2008} study the effect of noise on the reconstruction of
single fluorophores using Fourkas' expressions. Their main results are
unaffected by the correction introduced in these notes, but because these notes
predict a higher intensity than Fourkas, Lu and Bout overpredict the effect of
noise in their model.

As mentioned after equation \ref{eq:approx}, Fourkas uses an approximation when
he directly maps the GJV in the back focal plane to the GJV in the detector
plane. To conserve energy, the power carried by a ray that makes an angle
$\theta$ with the optical axis needs to be attenuated by a factor of
$\sqrt{\cos\theta}$ \cite{nov, backer}. This attenuation factor can be ignored
under the paraxial approximation ($\cos\theta \approx 1$), which means that all
of the results in Fourkas and the correction in these notes only apply to low NA
lenses. Analytically extending these results to high NA lenses could be the
subject of future work.

As noted in \cite{lieb}, Fourkas' model only considers single molecules in
homogeneous environments. With an extension of the Green's tensor (equation
\ref{eq:greena}) \cite{nov, backer}, we could consider single molecules near
glass interfaces.

\bibliography{report}{}
\bibliographystyle{unsrt}

\end{document}

