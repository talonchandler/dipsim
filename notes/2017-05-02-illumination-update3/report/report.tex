\documentclass[11pt]{article}

%%%%%%%%%%%%
% Packages %
%%%%%%%%%%%%
\hyphenpenalty=10000
\usepackage{tocloft}
\renewcommand\cftsecleader{\cftdotfill{\cftdotsep}}
\def\undertilde#1{\mathord{\vtop{\ialign{##\crcr
$\hfil\displaystyle{#1}\hfil$\crcr\noalign{\kern1.5pt\nointerlineskip}
$\hfil\tilde{}\hfil$\crcr\noalign{\kern1.5pt}}}}}
\usepackage{xcolor}
\usepackage{hyperref}
\usepackage{epstopdf}
\usepackage{braket}
\usepackage{upgreek}
\usepackage{caption}
\usepackage{booktabs}
\usepackage{subcaption}
\usepackage{amssymb,latexsym,amsmath,gensymb}
\usepackage{latexsym}
\usepackage{graphicx}
\usepackage{float}
\usepackage{enumitem}
\usepackage{pdflscape}
\usepackage{url}
\usepackage{tikz, calc}
\usetikzlibrary{shapes.geometric, arrows, calc}
\tikzstyle{norm} = [rectangle, rounded corners, minimum width=2cm, minimum height=1cm,text centered, draw=black]
\tikzstyle{arrow} = [thick, ->, >=stealth]

\providecommand{\e}[1]{\ensuremath{\times 10^{#1}}} 
\providecommand{\mb}[1]{\mathbf{#1}}
\providecommand{\mh}[1]{\mathbf{\hat{#1}}}
\providecommand{\bs}[1]{\boldsymbol{#1}} 
\providecommand{\intinf}{\int_{-\infty}^{\infty}}
\providecommand{\fig}[4]{
  % filename, width, caption, label
\begin{figure}[h]
 \captionsetup{width=1.0\linewidth}
 \centering
 \includegraphics[width = #2\textwidth]{#1}
 \caption{#3}
 \label{fig:#4}
\end{figure}
}

\newcommand{\tensor}[1]{\overset{\text{\tiny$\leftrightarrow$}}{\mb{#1}}}
\newcommand{\tunderbrace}[2]{\underbrace{#1}_{\textstyle#2}}
\providecommand{\figs}[7]{
  % filename1, filename2, caption1, caption2, label1, label2, shift
\begin{figure}[H]
\centering
\begin{minipage}[b]{.45\textwidth}
  \centering
  \includegraphics[width=1.0\linewidth]{#1}
  \captionsetup{justification=justified, singlelinecheck=true}
  \caption{#3}
  \label{fig:#5}
\end{minipage}
\hspace{2em}
\begin{minipage}[b]{.45\textwidth}
  \centering
  \includegraphics[width=1.0\linewidth]{#2}
  \vspace{#7em}
  \captionsetup{justification=justified}
  \caption{#4}
  \label{fig:#6}
\end{minipage}
\end{figure}
}
\makeatletter

\providecommand{\code}[1]{
\begin{center}
\lstinputlisting{#1}
\end{center}
}

%%%%%%%%%%%
% Spacing %
%%%%%%%%%%%
% Margins
\usepackage[
top    = 1.5cm,
bottom = 1.5cm,
left   = 1.5cm,
right  = 1.5cm]{geometry}

% Indents, paragraph space
\usepackage{parskip} 

% Section spacing
\usepackage{titlesec}
\titlespacing*{\title}
{0pt}{0ex}{0ex}
\titlespacing*{\section}
{0pt}{0ex}{0ex}
\titlespacing*{\subsection}
{0pt}{0ex}{0ex}
\titlespacing*{\subsubsection}
{0pt}{0ex}{0ex}

% Line spacing
\linespread{1.1}

%%%%%%%%%%%%
% Document %
%%%%%%%%%%%%
\begin{document}
\title{\vspace{-2.5em}Estimating A Single Molecule's Orientation With Polarized
  Illumination Microscopy\vspace{-2em}} \author{}% and Patrick La Rivi\`ere}
\date{\vspace{-1em}May 2, 2017\vspace{-1em}}
\maketitle

\section{Introduction}
In this work we evaluate wide-field polarized microscope designs for determining
the orientation of single molecules. In section 2 we model the relationship
between a molecule's orientation, the microscope geometry, and the measured
intensity. In section 3 we present simulation results to show how these
microscopes behave, and we compare the ability of several microscope designs to
reconstruct the orientation of single dipoles using techniques from estimation
theory. Finally, in sections 4 and 5 we summarize our results and present plans
for future work.

\section{Methods}
In this section we model the relationship between a single molecule's
orientation, the microscope's geometry, and the measured intensity. We split the
model into two parts. First, we develop the illumination model and calculate the
excitation efficiency, $\eta_{\text{exc}}$---the fraction of the maximum power
absorbed by the molecule. Next, we develop the detection model and calculate the
detection efficiency, $\eta_{\text{det}}$---the fraction of the power emitted by
the molecule that we detect. Finally, we multiply the excitation efficiency and
the detection efficiency to find the total efficiency, $\eta_{\text{tot}}$---a
quantity that is proportional to the measured intensity. 

\subsection{Illumination Model}
In this section we model the excitation efficiency of a single molecule with absorption
dipole moment $\hat{\bs{\mu}}_{\text{abs}}$ at the focal point of a
condenser. We excite the molecule with K\"{o}hler illumination with a linear
polarizer in the back focal plane of the condenser. Each point in the back focal
plane illuminates the molecule with a polarized plane wave rotated
by the condenser, and each polarized plane wave acts on the molecule independently, so
we can find the excitation efficiency by integrating over the back focal plane of the condenser. Therefore, the excitation efficiency is
\begin{align}
  \eta_{\text{exc}} &= \frac{\int_{\text{bfp}} d\mathbf{r}'|\hat{\bs{\mu}}_{\text{abs}}^{\dagger}\mathbf{\hat{A}}_{\text{ffp}}|^2}{\int_{\text{bfp}}d\mathbf{r}'}\label{eq:illum}\\
\mathbf{\hat{A}}_{\text{ffp}}(\mh{r}') &= \tilde{\mb{R}}(\mh{r}')\mathbf{A}_{\text{bfp}} 
\end{align}
where $\mh{A}_{\text{bfp}}$ is the direction of the polarizer in the back focal
plane, $\mh{r}'$ is the position coordinate in the back focal plane,
$\tilde{\mb{R}}(\mh{r}')$ is the position-dependent rotation matrix that rotates
the generalized Jones vector (GJV) in the back focal to a GJV in the front focal
plane, $\mh{A}_{\text{ffp}}$ is the GJV in the front focal plane due to the
point $\mh{r}'$ in the back focal plane, $\hat{\bs{\mu}}_{\text{abs}}$ is the
molecule's absorption dipole moment, and ${}^{\dagger}$ denotes the adjoint
operator.

$\eta_{\text{exc}}$ is normalized by the area of the back focal plane so the it
can be interpreted as the fraction of the maximum power absorbed by the
molecule. If the back focal plane is a pinhole then the molecule is illuminated
by a single plane wave. If the plane wave is aligned with the dipole
($\mathbf{\hat{A}}_{\text{bfp}} = \hat{\bs{\mu}}_{\text{abs}}$) then
$\eta_{\text{exc}}=1$, the most power the dipole can absorb. Opening the
aperture or rotating the molecule will decrease the excitation efficiency.

Equation \ref{eq:illum} is too computationally expensive to calculate for each
illumination and dipole orientation. Instead, we follow \cite{backer} and factor
the integral so that we can calculate an illumination basis once for each
illumination geometry. The factored form of equation \ref{eq:illum} is
\begin{align}
  \eta_{\text{exc}} &= \frac{1}{\int_{\text{bfp}}d\mathbf{r}'}[\mu_x^2, \mu_y^2, \mu_z^2, \mu_x\mu_y, \mu_x\mu_z, \mu_y\mu_z]\cdot\int_{\text{bfp}} d\mathbf{r}'[|A_x|^2, |A_y|^2, |A_z|^2, 2\text{Re}\{A_x^*A_y\}, 2\text{Re}\{A_x^*A_z\}, 2\text{Re}\{A_y^*A_z\}]^T \label{eq:factored}
\end{align}
where
\begin{align}
  \mb{\hat{A}}_{\text{ffp}} &= A_x\mh{i} + A_y\mh{j} + A_z\mh{k}\\
  \hat{\bs{\mu}}_{\text{abs}} &= \mu_x\mh{i} + \mu_y\mh{j} + \mu_z\mh{k}
\end{align}
We plot the illumination basis elements in figure \ref{fig:illum_basis}. 

\fig{../figures/illum_basis.png}{1.0}{Illumination basis elements as a function of
  polarization orientation and back focal plane radius. Row 1: schematics of the
  illumination geometries---a polarizer is placed in the back focal plane of the
  condenser lens. Row 2 and 3: plots of the illumination basis elements as a
  function of $\rho/f$---the ratio of the back focal plane radius to the focal
  length. The illumination basis elements should be interpreted with
  equation \ref{eq:factored} in view---they do not have a direct physical
  interpretation. Columns: varying polarization orientations. }{illum_basis}

\subsection{Detection Model}
In this section we model the detection efficiency---the fraction of the power
emitted by a single molecule that we detect. We follow Fourkas \cite{fourkas} and find that the
detection efficiency is
\begin{align}
  \eta_{\text{det}} &= 2\left(A + B\sin^2\theta\right)\\
  A &= \frac{1}{4} - \frac{3}{8} \cos\alpha + \frac{1}{8} \cos^{3}\alpha\\
  B &= \frac{3}{16} \cos\alpha - \frac{3}{16} \cos^{3}\alpha
\end{align}
where $\theta$ is the angle between the molecule's emission dipole moment
$\hat{\bs{\mu}}_{\text{em}}$ and the detection arm's optical axis, and $\alpha$
is the half angle of the collection cone. See the note set ``Inconsistency in
rapid determination of the three-dimensional orientation of single molecules''
for the derivation of this model. Notice that we're using the corrected versions
of Fourkas' variables $A$ and $B$.

For this work we assume that $\hat{\bs{\mu}}_{\text{abs}} = \hat{\bs{\mu}}_{\text{em}}$.

\section{Results}
\subsection{Excitation, Detection, and Total Efficiency}
Figure \ref{fig:efficiency} shows representative results of the excitation,
detection, and total efficiencies as a function of dipole direction and
microscope geometry. Notice that a single frame does not give us enough
information to recover a dipole's direction. In the next section we will
consider using the information from combinations of single frame microscopes
like the ones shown in Figure \ref{fig:efficiency}.

\fig{../figures/geometry.png}{1.0}{Representative examples of single frame microscopes. Row
  1: schematics of the illumination and detection geometries. Rows 2-4: excitation, detection, and total efficiency as a function of dipole direction. The total efficiency is the product of the excitation efficiency and the detection efficiency.}{efficiency}

\subsection{Dipole Orientation Estimates With Multiple Frame Experiments}
In this section we combine the results from several single frame microscopes to
create multiple frame experiments, and we investigate the possibility of using
the information from multiple frames to reconstruct a single molecule's
orientation. To assess our ability to reconstruct a molecule's orientation from
intensity measurements, we parameterize the molecule's orientation with
spherical coordinates
($\hat{\bs{\mu}}_{\text{em}} = \sin\Theta\cos\Phi\ \hat{\mb{i}} +
\sin\Theta\sin\Phi\ \hat{\mb{j}} + \cos\Theta\ \hat{\mb{k}}$), find the minimum
variance of an unbiased estimator for these parameters ($\sigma^2_{\Theta}$ and
$\sigma^2_{\Phi}$) using the Cramer Rao lower bound, and calculate a metric we
call the solid-angle uncertainty---
$\sigma_{\Omega} = \sin\Theta\sigma_{\Theta}\sigma_{\Phi}$. We use
$\sigma_{\Omega}$ as our figure of merit because it is easy to interpret---given
a single molecule's orientation and a set of measurements, $\sigma_{\Omega}$ is
the solid angle of the cone of uncertainty about the molecule's reconstructed
orientation.

\subsection{Single-View Microscope Results}
Figure \ref{fig:one_arm} shows the solid-angle uncertainty for multiple-frame,
single-view experiments. Each experiment consists of four frames with different
polarization orientations and fixed illumination and detection geometry. We used
a Poisson noise model with 1000 illuminating photons per frame, a back focal
plane radius of 3/10 the focal length of the condenser, and a detection NA of
1.3.
  
\fig{../figures/singlearm3.png}{0.9}{Solid-angle uncertainty for single view multiple frame
  microscopes. Row 1: schematics of the microscope geometries---we detect four
  frames with different illumination polarizations. Row 2: solid-angle
  uncertainty as a function of dipole orientation. A small $\sigma_{\Omega}$
  indicates that the orientation of a dipole oriented along this direction can
  be precisely recovered from the data collected with this microscope
  geometry. Columns: varying angles between the illumination and detection
  arms. }{one_arm}

\subsection{Dual-View Microscope Results}

Figure \ref{fig:dual_arm} shows the solid-angle uncertainty for multiple-frame,
dual-view experiments. We've repeated the experiments in the previous section,
but now we collect four polarization frames with one view, swap the illumination
and detection arms, then collect another four polarization frames. We halved the
number of illuminating photons to 500 per frame to allow for a fair comparison
between the single arm and dual arm results.

\fig{../figures/dualarm3.png}{0.9}{Solid-angle uncertainty for dual view multiple frame
  microscopes. Same caption as Figure \ref{fig:one_arm} except here we
  illuminate and detect along both arms. Notice that the 45${}^{\circ}$
  detection scheme provides the most uniform solid-angle uncertainty of the
  designs considered here.}{dual_arm}

\section{Discussion}
An ideal microscope geometry would allow us to reconstruct the orientation of a
single molecule with a constant solid-angle uncertainty for every orientation of
the molecule. We can't accomplish this goal without a large number of detectors
at all positions around the dipole, so instead we choose microscope geometries
that are experimentally feasible and give us the most uniform solid-angle
uncertainty.

Figure \ref{fig:one_arm} shows that single-view microscopes cannot be used to
reconstruct single molecules with uniform solid-angle uncertainty. All
single-view geometries are degenerate near a plane of molecule orientations. In
the epi-detection case the degeneracy occurs near the plane orthogonal to the
optical axis. A molecule lying near this plane will give rise to data that
cannot be used to determine whether the molecule lies above or below the
plane. Lu discussed this degeneracy in detail and developed techniques to avoid
reconstruction issues \cite{lux}. When the detector is rotated the plane of
degeneracy moves but does not disappear. Notice that moving the detector away
from epi-detection always decreases the detection efficiency, so there is no
advantage to using oblique or orthogonal detection geometries with only one
view.

Figure \ref{fig:dual_arm} shows that oblique and orthogonal geometries can be
advantageous for dual-view microscopes. The first column of Figure
\ref{fig:dual_arm} shows the epi-detection case with the plane of degeneracy
orthogonal to the detection axis. The second and third columns show that
rotating the second arm away from the first arm creates a dramatically more
uniform solid-angle uncertainty. Between the epi-, 45${}^{\circ}$-, and
ortho-detection geometries, we can see that the 45${}^{\circ}$ detection
geometry creates the most uniform solid-angle uncertainty. We note that the
results extend symmetrically past angles larger than 90${}^{\circ}$. For
example, 135${}^{\circ}$ detection gives equivalent results to 45${}^{\circ}$
detection, and 135${}^{\circ}$ detection may be more experimentally feasible.

\section{Next Steps}
Limitations and/or next steps:
\begin{enumerate}
\item Both the illumination and detection models use the paraxial approximation. I know how to extend to the high-NA case, but the conclusions in this work do not depend
  on high-NA lenses. Do you think I should extend now or later?
\item I'm currently only using a Poisson noise model, but I've implemented a
  Poisson + Gaussian model as well if we'd like results with read-out noise. Read-out
  noise won't cause any major difference---it will just raise the solid-angle
  uncertainty baseline.
\item Currently I'm calculating Equation \ref{eq:factored} numerically, and I'm
  getting reasonable results (see Figure \ref{fig:illum_basis} and Figure
  \ref{fig:efficiency}). I can write \ref{eq:factored} in terms of six
  integrals, and I think that those integrals will simplify under the paraxial
  assumption. This may allow us to consider the effect of illumination in closed
  form. Still investigating.
\item The 45${}^{\circ}$ and 135${}^{\circ}$ cases are identical if the molecule
  is in a homogeneous environment. I have the tools to consider the case where
  the molecule is near an interface \cite{backer}, but I haven't implemented
  this yet.
\end{enumerate}

Questions:
\begin{enumerate}
\item I've parameterized the radius of the back focal plane with $\rho/f$, the
  ratio the back focal plane radius to the focal length. Would NA${}_{\text{illum}}$
  be more appropriate?
\item Is 500/1000 illuminating photons per frame appropriate? What are typical
  read out noise levels?
\end{enumerate}

\bibliography{report}{}
\bibliographystyle{unsrt}

\end{document}

