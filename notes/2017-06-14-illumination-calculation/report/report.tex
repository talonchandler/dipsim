\documentclass[11pt]{article}

%%%%%%%%%%%%
% Packages %
%%%%%%%%%%%%
\hyphenpenalty=10000
\usepackage{tocloft}
\renewcommand\cftsecleader{\cftdotfill{\cftdotsep}}
\def\undertilde#1{\mathord{\vtop{\ialign{##\crcr
$\hfil\displaystyle{#1}\hfil$\crcr\noalign{\kern1.5pt\nointerlineskip}
$\hfil\tilde{}\hfil$\crcr\noalign{\kern1.5pt}}}}}
\usepackage{cleveref}
\usepackage{xcolor}
\usepackage{hyperref}
\usepackage{epstopdf}
\usepackage{braket}
\usepackage{upgreek}
\usepackage{caption}
\usepackage{booktabs}
\usepackage{subcaption}
\usepackage{amssymb,latexsym,amsmath,gensymb}
\usepackage{latexsym}
\usepackage{graphicx}
\usepackage{float}
\usepackage{enumitem}
\usepackage{pdflscape}
\usepackage{url}
\usepackage{tikz, calc}
\usetikzlibrary{shapes.geometric, arrows, calc}
\tikzstyle{norm} = [rectangle, rounded corners, minimum width=2cm, minimum height=1cm,text centered, draw=black]
\tikzstyle{arrow} = [thick, ->, >=stealth]

\providecommand{\e}[1]{\ensuremath{\times 10^{#1}}} 
\providecommand{\mb}[1]{\mathbf{#1}}
\providecommand{\mh}[1]{\mathbf{\hat{#1}}}
\providecommand{\bs}[1]{\boldsymbol{#1}} 
\providecommand{\intinf}{\int_{-\infty}^{\infty}}
\providecommand{\fig}[4]{
  % filename, width, caption, label
\begin{figure}[h]
 \captionsetup{width=1.0\linewidth}
 \centering
 \includegraphics[width = #2\textwidth]{#1}
 \caption{#3}
 \label{fig:#4}
\end{figure}
}

\newcommand{\tensor}[1]{\overset{\text{\tiny$\leftrightarrow$}}{\mb{#1}}}
\newcommand{\tunderbrace}[2]{\underbrace{#1}_{\textstyle#2}}
\providecommand{\figs}[7]{
  % filename1, filename2, caption1, caption2, label1, label2, shift
\begin{figure}[H]
\centering
\begin{minipage}[b]{.45\textwidth}
  \centering
  \includegraphics[width=1.0\linewidth]{#1}
  \captionsetup{justification=justified, singlelinecheck=true}
  \caption{#3}
  \label{fig:#5}
\end{minipage}
\hspace{2em}
\begin{minipage}[b]{.45\textwidth}
  \centering
  \includegraphics[width=1.0\linewidth]{#2}
  \vspace{#7em}
  \captionsetup{justification=justified}
  \caption{#4}
  \label{fig:#6}
\end{minipage}
\end{figure}
}
\makeatletter

\providecommand{\code}[1]{
\begin{center}
\lstinputlisting{#1}
\end{center}
}

\newcommand{\crefrangeconjunction}{--}
%%%%%%%%%%%
% Spacing %
%%%%%%%%%%%
% Margins
\usepackage[
top    = 1.5cm,
bottom = 1.5cm,
left   = 1.5cm,
right  = 1.5cm]{geometry}

% Indents, paragraph space
\usepackage{parskip} 

% Section spacing
\usepackage{titlesec}
\titlespacing*{\title}
{0pt}{0ex}{0ex}
\titlespacing*{\section}
{0pt}{0ex}{0ex}
\titlespacing*{\subsection}
{0pt}{0ex}{0ex}
\titlespacing*{\subsubsection}
{0pt}{0ex}{0ex}

% Line spacing
\linespread{1.1}

%%%%%%%%%%%%
% Document %
%%%%%%%%%%%%
\begin{document}
\title{\vspace{-2.5em} Closed Form Expressions for the Excitation and Detection Efficiency of Single Dipoles Under Polarized Wide-Field Illumination \vspace{-1em}}
\author{Talon Chandler}% and Patrick La Rivi\`ere}
\date{\vspace{-1em}\today\vspace{-1em}}
\maketitle
\section{Introduction}
In these notes I will develop the relationships between the excitation/detection
efficiencies and the dipole orientation, polarizer orientation, and microscope
geometry for wide-field illumination microscopes. First, I will write out the
relationships for the epi-illumination and epi-detection case, then I will
generalize the relationships for oblique or orthogonal geometries. 

\section{Detection Efficiency}
In the 2017-04-25 notes I followed Fourkas and showed that the detection
efficiency (the fraction of emitted power that we collect) for a polarized
wide-field microscope is
\begin{align}
  \eta_{\text{det},\phi_{\text{det}}}  = \frac{\int_{\Omega}d\mh{r}\ \left|\mh{P}_{\text{det}}\cdot\tilde{\mb{R}}(\mh{r})(\mh{r}\times\hat{\bs{\mu}}_{\text{em}}\times\mh{r})\right|^2}{\int_{\mathbb{S}^2}d\mh{r}\left|\mh{r}\times\hat{\bs{\mu}}_{\text{em}}\times\mh{r}\right|^2}\label{eq:final}
\end{align}
where
\begin{align}
  \hat{\mb{r}} &= \sin\theta\cos\phi\hat{\mb{i}} + \sin\theta\sin\phi\hat{\mb{j}} + \cos\theta\hat{\mb{k}}\label{eq:r_coords}\ \ \ \text{is the dummy integration vector,}\\
  \hat{\bs{\mu}}_{\text{em}} &= \sin\Theta\cos\Phi\hat{\mb{i}} + \sin\Theta\sin\Phi\hat{\mb{j}} + \cos\Theta\hat{\mb{k}}\ \ \ \text{is the emission dipole moment,}\label{eq:mu_coords}\\
  \tilde{\mb{R}}(\mh{r}) &= \begin{bmatrix} \cos\theta\cos^2\phi + \sin^2\phi & (\cos\theta -1)\sin\phi\cos\phi & -\sin\theta\cos\phi\\ (\cos\theta - 1)\sin\phi\cos\phi & \cos\theta\sin^2\phi + \cos^2\phi & -\sin\theta\sin\phi \\ \sin\theta\cos\phi& \sin\theta\sin\phi & \cos\theta \end{bmatrix}\label{eq:matrix}\\&\text{is the rotation matrix that models the objective lens,}\\
  \hat{\mb{P}}_{\text{det}} &= \cos\phi_{\text{det}}\hat{\mb{i}} + \sin\phi_{\text{det}}\hat{\mb{j}}\ \ \ \text{is the pass axis of the polarizer,}\\
  \Omega &= \{(\phi, \theta)|\phi \in (0,2\pi]\ \text{and}\ \theta\in (0, \alpha]\}\ \ \ \text{is the cap of the sphere collected by the objective.}\label{eq:omega}
\end{align}
Plugging equations \ref{eq:r_coords}--\ref{eq:omega} into equation \ref{eq:final} and
simplifying gives Fourkas' result with an extra factor of $\frac{3}{2}$
\begin{subequations}
\begin{align}
  \eta_{\text{det},0}(\Theta, \Phi, \alpha) &= A + B\sin^{2}{\Theta} + C\sin^{2}{\Theta} \cos{2 \Phi}\\
  \eta_{\text{det},45}(\Theta, \Phi, \alpha) &= A + B\sin^{2}{\Theta} + C\sin^{2}{\Theta} \sin{2 \Phi}\\
  \eta_{\text{det},90}(\Theta, \Phi, \alpha) &= A + B\sin^{2}{\Theta} - C\sin^{2}{\Theta} \cos{2 \Phi}\\
  \eta_{\text{det},135}(\Theta, \Phi, \alpha) &= A + B\sin^{2}{\Theta} - C\sin^{2}{\Theta} \sin{2 \Phi}
\end{align}\label{eq:int1}
\end{subequations}
where
\vspace{-1em}
\begin{subequations}
\begin{align}
  A &= \frac{1}{4} - \frac{3}{8} \cos{\alpha } + \frac{1}{8} \cos^{3}{\alpha }\\
  B &= \frac{3}{16} \cos{\alpha } - \frac{3}{16} \cos^{3}{\alpha }\\
  C &= \frac{7}{32} - \frac{3}{32} \cos{\alpha } - \frac{3}{32} \cos^{2}{\alpha } - \frac{1}{32} \cos^{3}{\alpha}.
\end{align}\label{eq:coeff2}%
\end{subequations}
If we remove the polarizer from the detection arm the detection efficiency is
\begin{align}
  \eta_{\text{det},\times}(\Theta, \alpha) = \eta_{\text{det},0} + \eta_{\text{det},90} = 2A + 2B\sin^2\Theta
\end{align}
where $\times$ denotes ``no polarizer''.

\section{Excitation Efficiency}
Here we calculate the excitation efficiency (the fraction of incident power that
excites the fluorophore). We calculate the power that excites the dipole by
taking the 3D Jones vector that is incident on the condenser
$\mh{P}_{\text{exc}}$, pass it through the condenser using a rotation matrix
$\tilde{\mb{R}}$, take the dot product with the absorption dipole moment
$\hat{\bs{\mu}}_{\text{abs}}$, take the modulus squared to find the intensity,
then integrate over the cone of illumination. Finally, we find the excitation
efficiency by dividing by the total power incident on the fluorophore. The final
expression is
\begin{align}
  \eta_{\text{exc},\phi_{\text{exc}}} = \frac{\int_{\Omega}d\mh{r}|\hat{\bs{\mu}}_{\text{abs}}\cdot\tilde{\mb{R}}(\mh{r})\mh{P}_{\text{exc}}|^2}{\int_{\Omega}d\mh{r}}\label{eq:finalexc}
\end{align}
where
\begin{align}
  \hat{\mb{r}} &= \sin\theta\cos\phi\hat{\mb{i}} + \sin\theta\sin\phi\hat{\mb{j}} + \cos\theta\hat{\mb{k}}\label{eq:r_coords2}\ \ \ \text{is the dummy integration vector,}\\
  \hat{\bs{\mu}}_{\text{abs}} &= \sin\Theta\cos\Phi\hat{\mb{i}} + \sin\Theta\sin\Phi\hat{\mb{j}} + \cos\Theta\hat{\mb{k}}\ \ \ \text{is the absorption dipole moment,}\label{eq:mu_coords}\\
  \tilde{\mb{R}}(\mh{r}) &= \begin{bmatrix} \cos\theta\cos^2\phi + \sin^2\phi & (\cos\theta -1)\sin\phi\cos\phi & -\sin\theta\cos\phi\\ (\cos\theta - 1)\sin\phi\cos\phi & \cos\theta\sin^2\phi + \cos^2\phi & -\sin\theta\sin\phi \\ \sin\theta\cos\phi& \sin\theta\sin\phi & \cos\theta \end{bmatrix}\label{eq:matrix}\\&\text{is the rotation matrix that models the condenser lens,}\\
  \hat{\mb{P}}_{\text{exc}} &= \cos\phi_{\text{exc}}\hat{\mb{i}} + \sin\phi_{\text{det}}\hat{\mb{j}}\ \ \ \text{is the pass axis of the excitation polarizer,}\\
  \Omega &= \{(\phi, \theta)|\phi \in (0,2\pi]\ \text{and}\ \theta\in (0, \alpha]\}\ \ \ \text{is the cap of the sphere illuminated by the condenser.}\label{eq:omega2}
\end{align}
Plugging equations \ref{eq:r_coords2}--\ref{eq:omega2} into equation
\ref{eq:finalexc} and simplifying gives
\begin{subequations}
\begin{align}
  \eta_{\text{exc},0}(\Theta, \Phi, \alpha) &= D(A + B\sin^{2}{\Theta} + C\sin^{2}{\Theta} \cos{2 \Phi})\\
  \eta_{\text{exc},45}(\Theta, \Phi, \alpha) &= D(A + B\sin^{2}{\Theta} + C\sin^{2}{\Theta} \sin{2 \Phi})\\
  \eta_{\text{exc},90}(\Theta, \Phi, \alpha) &= D(A + B\sin^{2}{\Theta} - C\sin^{2}{\Theta} \cos{2 \Phi})\\
  \eta_{\text{exc},135}(\Theta, \Phi, \alpha) &= D(A + B\sin^{2}{\Theta} - C\sin^{2}{\Theta} \sin{2 \Phi})
\end{align}\label{eq:int2}
\end{subequations}
where
\vspace{-1em}
\begin{align}
  D = \frac{4}{3(1 - \cos\alpha)}
\end{align}\label{eq:coeff2}%
The excitation efficiency is the same as the detection efficiency with an extra
factor $D$.

Finally, we'll confirm that the excitation efficiency reduces to Malus' law when $\Theta = \pi/2$ and as $\alpha \rightarrow 0$.
\begin{align*}
  &=\lim_{\alpha\rightarrow 0} \eta_{\text{exc},0}(\pi/2, \Phi, \alpha)\\
  &= \lim_{\alpha\rightarrow 0}D(A + B + C\cos{2 \Phi})\\
  &= \lim_{\alpha\rightarrow 0} \left(\frac{4}{3(1-\cos\alpha)}\right)\left(\frac{1}{4} - \frac{3}{16} \cos{\alpha } - \frac{1}{16} \cos^{3}{\alpha } + \left(\frac{7}{32} - \frac{3}{32} \cos{\alpha } - \frac{3}{32} \cos^{2}{\alpha } - \frac{1}{32} \cos^{3}{\alpha}\right)\cos 2\Phi\right)\\
  &= \lim_{\alpha\rightarrow 0} \frac{\frac{1}{4} - \frac{1}{4} \cos{\alpha } - \frac{1}{12} \cos^{3}{\alpha } + \left(\frac{7}{24} - \frac{1}{8} \cos{\alpha } - \frac{1}{8} \cos^{2}{\alpha } - \frac{1}{24} \cos^{3}{\alpha}\right)\cos 2\Phi}{1 - \cos\alpha}\\
  \intertext{Applying L'Hospital's rule gives}
  &= \lim_{\alpha\rightarrow 0} \frac{\frac{1}{4} \sin{\alpha } + \frac{1}{4}\sin\alpha\cos^{2}{\alpha } + \left(\frac{1}{8} \sin{\alpha } + \frac{1}{4}\sin \alpha \cos{\alpha } + \frac{1}{8} \sin\alpha\cos^{2}{\alpha}\right)\cos 2\Phi}{\sin\alpha}\\
  &= \frac{1}{2} + \frac{1}{2}\cos2\Phi = \boxed{\cos^2{\Phi}}.
\end{align*}

\section{Epi-illumination and Epi-detection Forward Model}
If we excite with polarized illumination from above and detect without a
polarizer from the same direction the intensity we expect to collect is
\begin{align}
  I_{\phi_{\text{exc}}} &= I_{\text{tot}}\eta_{\text{exc},\phi_{\text{exc}}}\eta_{\text{det}, \times}
\end{align}
where $I_{\text{tot}}$ is the intensity collected from the fluorophore if we had
an experimental setup that had an excitation and detection efficiency of one.

If we collect frames with four polarization settings we get the following expressions
\begin{subequations}
\begin{align}
  I_{0}(\Theta, \Phi, \alpha) &= I_{\text{tot}}D(A + B\sin^{2}{\Theta} + C\sin^{2}{\Theta} \cos{2 \Phi})(2A + 2B\sin^2\Theta)\\
  I_{45}(\Theta, \Phi, \alpha) &= I_{\text{tot}}D(A + B\sin^{2}{\Theta} + C\sin^{2}{\Theta} \sin{2 \Phi})(2A + 2B\sin^2\Theta)\\
  I_{90}(\Theta, \Phi, \alpha) &= I_{\text{tot}}D(A + B\sin^{2}{\Theta} - C\sin^{2}{\Theta} \cos{2 \Phi})(2A + 2B\sin^2\Theta)\\
  I_{135}(\Theta, \Phi, \alpha) &= I_{\text{tot}}D(A + B\sin^{2}{\Theta} - C\sin^{2}{\Theta} \sin{2 \Phi})(2A + 2B\sin^2\Theta).
\end{align}\label{eq:int1}
\end{subequations}

\section{Oblique or Orthogonal Arms}
The expressions above assume that the optical axis is aligned with the $\mh{z}$
axis. If the optical axis is along a different axis, we could modify the
expressions by (1) changing the limits of integration in equations
\ref{eq:final} and \ref{eq:finalexc} and recalculating or (2) performing a
change of coordinates in equation \ref{eq:int2}. Approach (2) is much easier and
uses the change of spherical coordinates derived in the 2017-06-09 notes. The
main results are reproduced here
\begin{subequations}
\begin{align}
  \Theta' &= \arccos\left(\sin\psi\cos\Phi\sin\Theta + \cos\psi\cos\Theta\right)\label{eq:thetap}\\
  \Phi' &= \arccos\left(\frac{\cos\psi\cos\Phi\sin\Phi - \sin\psi\cos\Theta}{\sqrt{1 - (\sin\psi\cos\Phi\sin\Theta + \cos\psi\cos\Theta)^2}}\right)\label{eq:phip}
\end{align}\label{eq:solution}% 
\end{subequations}
\begin{subequations}
\begin{align}
  \Theta &= \arccos\left(-\sin\psi\cos\Phi'\sin\Theta' + \cos\psi\cos\Theta'\right)\\
  \Phi &= \arccos\left(\frac{\cos\psi\cos\Phi'\sin\Phi' + \sin\psi\cos\Theta'}{\sqrt{1 - (\sin\psi\cos\Phi'\sin\Theta' + \cos\psi\cos\Theta')^2}}\right)
\end{align}\label{eq:solutionp}% 
\end{subequations}

The forward model for an oblique detection or excitation arm can be found by plugging
equation \ref{eq:solutionp} into equation \ref{eq:int1}.

\end{document}

