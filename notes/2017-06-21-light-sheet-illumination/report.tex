\documentclass[11pt]{article}

\usepackage{myreport}
\usepackage[american]{babel}
\usepackage[backend=biber, sorting=none]{biblatex}
\addbibresource{report.bib}

\begin{document}
\title{\vspace{-2.5em} Excitation and Detection Efficiency Under Scanned Light
  Sheet Illumination \vspace{-2.5em}} \author{Talon Chandler}
\date{\vspace{-1em}\today\vspace{-1em}}
\maketitle
\section{Introduction}
In these notes I will calculate the excitation and detection efficiency of a
single fluorophore under polarized scanned light sheet illumination. I will
start by finding the transverse and longitudinal fields at all positions in a
Gaussian beam using reasonable approximations. Next I will calculate the
excitation efficiency of a single fluorophore as the beam is scanned across the
fluorophore. I will combine these results with the results from previous notes
to write the complete forward model. Finally I will discuss the approximations
we may need to simplify reconstruction.

\section{Transverse and Longitudinal Fields In A Scanned Gaussian Beam}
The complex spatial electric field of a paraxial Gaussian beam propagating along the
$\mh{z}$ axis is \cite{nov}
\begin{align}
  \mb{E}(x,y,z) &= \mb{A}\frac{w_0}{w(z)}\text{exp}\left\{{-\frac{(x^2+y^2)}{w^2(z)}}+i\left[kz - \eta(z) + \frac{k(x^2+y^2)}{2R(z)}\right]\right\}\label{eq:gauss}
\end{align}
where
\begin{alignat}{2}  
  \mb{A} &= \cos\phi_{\text{pol}}\mh{x} + \sin\phi_{\text{pol}}\mh{y}\qquad &&\text{is the input Jones vector},\\
  w_0 &\approx \frac{2n}{k(\text{NA})}\qquad &&\text{is the beam waist radius,}\\
  z_0 &= \frac{kw_0^2}{2} &&\text{is the Rayleigh range,}\\
  k &= \frac{2\pi n}{\lambda}\qquad &&\text{is the wave number,}\\
  w(z) &= w_0\sqrt{1+\frac{z^2}{z_0^2}}\qquad &&\text{is the beam radius,}\\  
  R(z) &= z\left(1+\frac{z_0^2}{z^2}\right)\qquad &&\text{is the wavefront radius,}\\
  \eta(z) &= \text{arctan}\left(\frac{z}{z_0}\right)\qquad &&\text{is the phase correction.}
\end{alignat}
Equation \ref{eq:gauss} uses the paraxial approximation, so it can only be used
for beams with a waist that is significantly larger than the reduced wavelength
($w_0 \gg \lambda/n$). Notice that under the paraxial approximation the beam is uniformly polarized in the transverse plane.

We would like to calculate the longitudinal component of a Gaussian beam when
the beam waist approaches the reduced wavelength ($w_0 > \lambda/n$). One
approach is to numerically evaluate the Richards-Wolf diffraction integral
\cite{richards, biobeam}. This approach is time consuming and too
accurate for our needs---we only want to model weak longitudinal
fields. Instead, I will follow Novotny et. al. \cite{nov} and use a longitudinal
correction to equation \ref{eq:gauss}.

As written, equation \ref{eq:gauss} doesn't satisfy Gauss' law
($\nabla \cdot \mb{E} = 0$), so we will add a longitudinal field to correct
it. If the input beam is polarized along the $\mh{x}$ axis
($\phi_{\text{pol}}=0$), then we can rearrange Gauss' law to relate the
longitudinal and transverse fields with
\begin{align}
  E_z(x, y, z) = - \int \left[\frac{\partial}{\partial x} E_x(x,y,z)\right]dz. 
\end{align}
Carnicer et al. \cite{carnicer} worked through this integral using the angular
spectrum representation and found that
\begin{align}
  E_z(x,y,z) = -i\frac{2x}{kw_0^2}E_x(x,y,z) = -i\frac{x}{z_0}E_x(x,y,z)\label{eq:longit}. 
\end{align}

Equation \ref{eq:longit} means that:
\begin{itemize}
\item There is no longitudinal polarization in the $\mh{y}-\mh{z}$ plane
  because $E_z(0,y,z) = 0$.
\item There are longitudinal fields lobes on both sides of the optical axis
  along the transverse polarization direction.
\item The longitudinal fields are $90^{\circ}$ out of phase with the transverse
  fields which means that the total field is elliptically polarized off axis.
\item The longitudinal field strength is proportional to
  $\lambda/w_0^2$---highly focused beams have the strongest longitudinal fields.
\end{itemize}

If the input is polarized along the $\mh{x}$ axis, the corrected 3D Jones vector is
\begin{align}
\mb{A}(x,y,z) = \mh{x}- i\frac{x}{z_0}\mh{z}.
\end{align}

If the input polarization is arbitrary then the corrected 3D Jones vector is
\begin{align}
  \mb{A}(x,y,z) = \cos\phi_{\text{pol}}\mh{x} + \sin\phi_{\text{pol}}\mh{y} - i\frac{x\cos\phi_{\text{pol}} + y\sin\phi_{\text{pol}}}{z_0}\mh{z}.
\end{align}

If the beam is scanned along the $\mh{y}$ axis with velecity $v$ then the time
dependent 3D Jones vector is
\begin{align}
  \mb{A}(x,y,z,t) = \cos\phi_{\text{pol}}\mh{x} + \sin\phi_{\text{pol}}\mh{y} - i\frac{x\cos\phi_{\text{pol}} + (y - vt)\sin\phi_{\text{pol}}}{z_0}\mh{z}\label{eq:scanned_j}
\end{align}
and the time dependent electric field is
\begin{align}
  \mb{E}(x,y,z,t) &= \mb{A}(x,y,z,t)\frac{w_0}{w(z)}\text{exp}\left\{{-\frac{(x^2+(y-vt)^2)}{w^2(z)}}+i\left[kz - \eta(z) + \frac{k(x^2+(y-vt)^2)}{2R(z)}\right]\right\}\label{eq:scanned_e}.
\end{align}
We will use equations \ref{eq:scanned_j} and \ref{eq:scanned_e} to calculate
the excitation efficiency of a single fluorophore. Notice that as the Rayleigh
range $z_0$ increases the longitudinal electric field decreases, so we can ignore the
longitudinal component when the beam is weakly focused.

\section{Scanned Beam Excitation Efficiency}
We define the excitation efficiency of a fluorophore as the fraction of 
incident power that excites the fluorophore. If a fluorophore with absorption
dipole moment $\bs{\mu}_{\text{abs}}$ is placed in a time-independent complex
electric field $\mb{E}$, then the excitation efficiency is given by
\begin{align}
  \eta_{\text{exc}}(x,y,z) = \frac{|\bs{\mu}_{\text{abs}} \cdot \mb{E}(x,y,z)|^2}{|\mb{E}(x,y,z) |^2}.
\end{align}
If the fluorophore is placed in the path of a scanned laser then the electric
field becomes time dependent. If the laser is scanned extremely fast we would
need to consider the coherence of the electric field, but we will only consider
slow scanning here to simplify the calculation. Specifically, we require that
$v \ll w_0/\tau_c$---the scan velocity must be much less than the beam width
divided by the coherence time. In this case the excitation efficiency is
\begin{align}
  \eta_{\text{exc}}(x,y,z) = \frac{\intinf|\bs{\mu}_{\text{abs}} \cdot \mb{E}(x,y,z,t)|^2dt}{\intinf|\mb{E}(x,y,z,t)|^2dt}\label{eq:excitation}
\end{align}
We plug equation \ref{eq:scanned} into equation \ref{eq:excitation}, express
$\bs{\mu}_{\text{abs}}$ in spherical coordinates
\begin{align}
  \bs{\mu}_{\text{abs}} = \cos\Phi\sin\Theta\mh{x} + \sin\Phi\sin\Theta\mh{y} + \cos\Theta\mh{z},
\end{align}
and evaluate the integrals (see Appendix for details) to write the
excitation efficiency as
\begin{align}
  \eta_{\text{exc}}(x,y,z) = \frac{\sin^2\Theta\cos^2(\Phi-\phi_{\text{pol}}) + \cos^2\Theta\frac{x^2\cos^2\phi_{\text{pol}} + \frac{1}{4}w^2(z)\sin^2\phi_{\text{pol}}}{z_0^2}}{1+\frac{x^2\cos^2\phi_{\text{pol}} + \frac{1}{4}w^2(z)\sin^2\phi_{\text{pol}}}{z_0^2}}.\label{eq:strong}
\end{align}
If the beam is weakly focused then we can ignore the longitudinal excitation and
the excitation efficiency simplifies to
\begin{align}
  \eta_{\text{exc}} = \sin^2\Theta\cos^2(\Phi-\phi_{\text{pol}}).
\end{align}

\section{Detection Efficiency}
If we detect fluorescence in wide-field mode with an orthogonal arm and no
polarizer then the detection efficiency is
\begin{align}
  \eta_{\text{det}} = 2A + 2B\sin^2\Theta'
\end{align}
where
\begin{align}
  A &= \frac{1}{4} - \frac{3}{8}\cos\alpha + \frac{1}{8}\cos^3\alpha,\\
  B &= \frac{3}{16}\cos\alpha - \frac{3}{16}\cos^3\alpha, 
\end{align}
$\alpha = \text{arcsin(NA}/n)$ is the detection cone half angle, and $\Theta'$
is the angle between the dipole axis and the detection optical axis. See the
2017-06-09 notes for the relationship between $\Theta'$ and $\Theta,\Phi$. See
\cite{fourkas} and the 2017-04-25 notes for the derivation of the detection
efficiencies and additional expression for detection arms that use a polarizer.

\section{Orientation Forward Model}
The detected intensity is proportional to the the product of the excitation and
detection efficiencies. Using a weakly focused excitation beam and an
unpolarized detection arm gives us the following model
\begin{align}
  I_{\phi_{\text{pol}}} &= I_{\text{tot}}\sin^2\Theta\cos^2(\Phi - \phi_{\text{pol}})(2A+2B\sin^2\Theta')\label{eq:forward}
\end{align}
where $I_{\text{tot}}$ is the intensity we would collect if we had an excitation and detection efficiency of 1.

\section{Discussion}
Equation \ref{eq:strong} shows that for strongly focused beams the excitation
efficiency is a function of position. This couples the orientation and location
of the fluorophore and complicates our reconstruction. For now we'll use only
weakly focused beams so that we can ignore the longitudinal component.

To ignore the longitudinal component, we require that the ratio of the beam
waist to the Rayleigh range be much less than 2 ($w/z_0 \ll 2$).  The excitation
beam in \cite{wu2013} satisfies this criteria
$w/z_0 = 1.2\ \mu\text{m}/9\ \mu\text{m} = 0.13$, but we'll need to be careful if
we want to use beams that are more strongly focused.

Under the weak focusing approximation the orientation and position of
fluorophores are decoupled. This will allow us to split the reconstruction into
two steps (1) estimate the position of the fluorophores using unpolarized frames
(or the sum of orthogonally polarized frames) with established reconstruction
techniques then (2) estimate the orientation or the fluorophores using polarized
frames and equation \ref{eq:forward}.

Note that we are working in a different regime than Agrawal et. al. \cite{agrawal}. They
are considering imaging systems with better resolution than ours, so the
position and orientation are coupled and must be estimated together. At our
resolution, the position and orientation are decoupled so we can estimate them
separately.

\section{References}
\setlength\biblabelsep{0.025\textwidth}
\printbibliography[heading=none]
\pagebreak

\section{Appendix}
We will evaluate the following integrals to find the excitation efficiency
\begin{align}
  \eta_{\text{exc}}(x,y,z) = \frac{\intinf|\bs{\mu}_{\text{abs}} \cdot \mb{E}(x,y,z,t)|^2dt}{\intinf|\mb{E}(x,y,z,t)|^2dt}\label{eq:excitation}
\end{align}
where
\begin{align}
  \bs{\mu}_{\text{abs}} &= \cos\Phi\sin\Theta\mh{x} + \sin\Phi\sin\Theta\mh{y} + \cos\Theta\mh{z}\\
  \mb{E}(x,y,z,t) &= \mb{A}(x,y,z,t)\frac{w_0}{w(z)}\text{exp}\left\{{-\frac{(x^2+(y-vt)^2)}{w^2(z)}}+i\left[kz - \eta(z) + \frac{k(x^2+(y-vt)^2)}{2R(z)}\right]\right\}\\
  \mb{A}(x,y,z,t) &= \cos\phi_{\text{pol}}\mh{x} + \sin\phi_{\text{pol}}\mh{y} - i\frac{x\cos\phi_{\text{pol}} + (y - vt)\sin\phi_{\text{pol}}}{z_0}\mh{z}.\label{eq:scanned_jones}
\end{align}

We'll need the following facts
\begin{align}
  \intinf e^{-ax^2}dx &= \sqrt{\frac{\pi}{a}}\\
  \intinf xe^{-ax^2}dx &= 0 \label{eq:odd}\\ 
  \intinf x^2e^{-ax^2}dx &= \frac{1}{2a}\sqrt{\frac{\pi}{a}}.
\end{align}

The numerator is
\begin{align}
  = &\intinf|\bs{\mu}_{\text{abs}} \cdot \mb{E}(x,y,z,t)|^2dt\\
  \begin{split}
  = &\intinf\Bigg |\left[\cos\Phi\sin\Theta\cos\phi_{\text{pol}} + \sin\Phi\sin\Theta\sin\phi_{\text{pol}} - i\cos\Theta\frac{x\cos\phi_{\text{pol}} + (y - vt)\sin\phi_{\text{pol}}}{z_0}\right]\\
    &\ \ \ \ \ \ \ \ \frac{w_0}{w(z)}\text{exp}\left\{{-\frac{(x^2+(y-vt)^2)}{w^2(z)}}+i\left[kz - \eta(z) + \frac{k(x^2+(y-vt)^2)}{2R(z)}\right]\right\}\Bigg |^2dt.
  \end{split}\\
  \intertext{After changing variables $y' = y - vt$ and moving constants outside the integral we get}
  = &\frac{w_0^2}{w^2(z)}e^{-\frac{2x^2}{w^2(z)}}\intinf\left|\left[\cos\Phi\sin\Theta\cos\phi_{\text{pol}} + \sin\Phi\sin\Theta\sin\phi_{\text{pol}} - i\cos\Theta\frac{x\cos\phi_{\text{pol}} + y'\sin\phi_{\text{pol}}}{z_0}\right]\right|^2e^{-\frac{2y'^2}{w^2(z)}}dy'.
 \intertext{After expanding the square brackets we get}
   = &\frac{w_0^2}{w^2(z)}e^{-\frac{2x^2}{w^2(z)}}\intinf \Bigg [(\cos\Phi\sin\Theta\cos\phi_{\text{pol}} + \sin\Phi\sin\Theta\sin\phi_{\text{pol}})^2 +\cos^2\Theta\frac{(x\cos\phi_{\text{pol}} +y'\sin\phi_{\text{pol}})^2}{z_0^2} \Bigg ]e^{-\frac{2y'^2}{w^2(z)}}dy'\\
   = &\frac{w_0^2}{w(z)}e^{-\frac{2x^2}{w^2(z)}}\sqrt{\frac{\pi}{2}}\Bigg [\sin^2\Theta\cos^2(\Phi - \phi_{\text{pol}})  + \cos^2\Theta\frac{x^2\cos^2\phi_{\text{pol}} + \frac{1}{4}w^2(z)\sin^2\phi_{\text{pol}}}{z_0^2}  \Bigg ].
\end{align}
The denominator is
\begin{align}
  = &\intinf|\mb{E}(x,y,z,t)|^2dt\\
  = &\frac{w_0^2}{w^2(z)}e^{-\frac{2x^2}{w^2(z)}}\intinf\left|\left[\cos\phi_{\text{pol}}\mh{x} + \sin\phi_{\text{pol}}\mh{y} - i\frac{x\cos\phi_{\text{pol}} + y'\sin\phi_{\text{pol}}}{z_0}\mh{z}\right]\right|^2e^{-\frac{2y'^2}{w^2(z)}}dy'\\
  = &\frac{w_0^2}{w^2(z)}e^{-\frac{2x^2}{w^2(z)}}\intinf\left[1 + \frac{(x\cos\phi_{\text{pol}} + y'\sin\phi_{\text{pol}})^2}{z^2_0}\right]e^{-\frac{2y'^2}{w^2(z)}}dy'\\
  = &\frac{w_0^2}{w(z)}e^{-\frac{2x^2}{w^2(z)}}\sqrt{\frac{\pi}{2}}\left[1 + \frac{x^2\cos^2\phi_{\text{pol}} + \frac{1}{4}w^2(z)\sin^2\phi_{\text{pol}}}{z^2_0}\right].
\end{align}
The final excitation efficiency is
\begin{align*}
\eta_{\text{exc}}(x,y,z) = \frac{\sin^2\Theta\cos^2(\Phi - \phi_{\text{\text{pol}}}) + \cos^2\Theta\frac{x^2\cos^2\phi_{\text{pol}} + \frac{1}{4}w^2(z)\sin^2\phi_{\text{pol}}}{z_0^2}}{1 + \frac{x^2\cos^2\phi_{\text{pol}} + \frac{1}{4}w^2(z)\sin^2\phi_{\text{pol}}}{z^2_0}}.
\end{align*}

\end{document}

