\documentclass[11pt]{article}

%%%%%%%%%%%%
% Packages %
%%%%%%%%%%%%
\hyphenpenalty=10000
\usepackage{array,multirow}
\usepackage{tocloft}
\renewcommand\cftsecleader{\cftdotfill{\cftdotsep}}
\def\undertilde#1{\mathord{\vtop{\ialign{##\crcr
$\hfil\displaystyle{#1}\hfil$\crcr\noalign{\kern1.5pt\nointerlineskip}
$\hfil\tilde{}\hfil$\crcr\noalign{\kern1.5pt}}}}}
\usepackage{cleveref}
\usepackage{xcolor}
\usepackage{hyperref}
\usepackage{epstopdf}
\usepackage{braket}
\usepackage{upgreek}
\usepackage{caption}
\usepackage{booktabs}
\usepackage{subcaption}
\usepackage{amssymb,latexsym,amsmath,gensymb}
\usepackage{latexsym}
\usepackage{graphicx}
\usepackage{float}
\usepackage{enumitem}
\usepackage{pdflscape}
\usepackage{url}
\usepackage{tikz, calc}
\usetikzlibrary{shapes.geometric, arrows, calc}
\tikzstyle{norm} = [rectangle, rounded corners, minimum width=2cm, minimum height=1cm,text centered, draw=black]
\tikzstyle{arrow} = [thick, ->, >=stealth]

\providecommand{\e}[1]{\ensuremath{\times 10^{#1}}} 
\providecommand{\mb}[1]{\mathbf{#1}}
\providecommand{\mh}[1]{\mathbf{\hat{#1}}}
\providecommand{\bs}[1]{\boldsymbol{#1}} 
\providecommand{\intinf}{\int_{-\infty}^{\infty}}
\providecommand{\fig}[4]{
  % filename, width, caption, label
\begin{figure}[h]
 \captionsetup{width=1.0\linewidth}
 \centering
 \includegraphics[width = #2\textwidth]{#1}
 \caption{#3}
 \label{fig:#4}
\end{figure}
}

\newcommand{\tensor}[1]{\overset{\text{\tiny$\leftrightarrow$}}{\mb{#1}}}
\newcommand{\tunderbrace}[2]{\underbrace{#1}_{\textstyle#2}}
\providecommand{\figs}[7]{
  % filename1, filename2, caption1, caption2, label1, label2, shift
\begin{figure}[H]
\centering
\begin{minipage}[b]{.45\textwidth}
  \centering
  \includegraphics[width=1.0\linewidth]{#1}
  \captionsetup{justification=justified, singlelinecheck=true}
  \caption{#3}
  \label{fig:#5}
\end{minipage}
\hspace{2em}
\begin{minipage}[b]{.45\textwidth}
  \centering
  \includegraphics[width=1.0\linewidth]{#2}
  \vspace{#7em}
  \captionsetup{justification=justified}
  \caption{#4}
  \label{fig:#6}
\end{minipage}
\end{figure}
}
\makeatletter

\providecommand{\code}[1]{
\begin{center}
\lstinputlisting{#1}
\end{center}
}

\newcommand{\crefrangeconjunction}{--}
%%%%%%%%%%%
% Spacing %
%%%%%%%%%%%
% Margins
\usepackage[
top    = 1.5cm,
bottom = 1.5cm,
left   = 1.5cm,
right  = 1.5cm]{geometry}

% Indents, paragraph space
%\usepackage{parskip} 
\setlength{\parskip}{1.5ex}

% Section spacing
\usepackage{titlesec}
\titlespacing*{\title}
{0pt}{0ex}{0ex}
\titlespacing*{\section}
{0pt}{0ex}{0ex}
\titlespacing*{\subsection}
{0pt}{0ex}{0ex}
\titlespacing*{\subsubsection}
{0pt}{0ex}{0ex}

% Line spacing
\linespread{1.1}

%%%%%%%%%%%%
% Document %
%%%%%%%%%%%%
\begin{document}
\title{\vspace{-2.5em} More Design Studies of Multiview Polarized \\Illumination
  and/or Detection Microscopy \vspace{-1em}} \author{Talon
  Chandler}% and Patrick La Rivi\`ere}
\date{\vspace{-1em}\today\vspace{-1em}}
\maketitle
\section{Introduction}
In these notes I will use the model and metrics we established in the working
paper draft to study three more design questions:
\begin{itemize}
\item What is the effect of using fewer than four illumination polarization orientation?
\item Do polarized illumination or polarized detection microscopes provide
  better orientation reconstructions?
\item How do microscopes with polarizers on both the illumination and detection
  paths compare to microscopes with polarizers on only the illumination or
  detection paths.
\end{itemize}

\section{Methods}
Figure \ref{fig:single-frame} shows the efficiencies for single measurements
taken with polarized illumination and/or detection microscopes. These results
were created with Fourkas' expressions on the detection side and my expressions
from the paper on the illumination side. Notice that the dashed arrows indicate
polarized detectors. Also notice that the detection efficiency range is only
0-0.25 instead of 0-0.5 in the paper because the polarizer on the detector
blocks half of the photons.

Unless otherwise stated, all of the designs I will compare in these notes will
use NA${}_{\text{ill}} = 0$ and NA${}_{\text{det}} = 0.8$, and all 2-view
microscopes will use orthogonal views.

All designs use the same total sample exposure. This means that
$I_{\text{exp}} = 4000/N$ where $N$ is the number of intensity
measurements. Note that $N$ includes polarized detection orientations as well as
polarized illumination orientations.

\fig{../figures/single-frame.pdf}{1.0}{Representative examples of single
  intensity measurements. Black dots indicate where the Cartesian unit vectors
  intersect the unit sphere. \newline \newline \textbf{Columns left to right:}
  1) schematics where the solid line encloses the illumination solid angle, the
  dashed line encloses the detection solid angle, the solid arrow indicates the
  transmission axis of the illumination polarizer, and the dashed arrow
  indicates the transmission axis of the detection polarizer; 2) the absorption
  efficiency; 3) the detection efficiency; 4) the total efficiency, the product
  of the absorption and detection efficiencies.\newline \newline \textbf{Rows
    top to bottom:} 1) coincident illumination ($\text{NA} = 0.8$ with
  $x$-polarized light) and detection (NA = 1.1); 2) non-coincident orthogonal
  illumination ($\text{NA} = 0.8$) and detection ($\text{NA} = 0.8$); 3)
  non-coincident 135${}^{\circ}$-separated illumination ($\text{NA} = 0.8$) and
  detection ($\text{NA} = 0.8$). All simulations use $n=1.33$.}{single-frame}

\section{Results \& Discussion}
Table 1 shows the results for microscopes with varying polarization on the
detection arm, the illumination arm, and both arms; varying numbers of views;
and varying numbers of polarization orientations per view.


\fig{../figures/compare-illumination.pdf}{0.6}{Comparing polarized illumination
  microscope designs. Each row in this figure corresponds to a row in the top
  third of Table 1.}{compare-illumination}
\fig{../figures/compare-detection.pdf}{0.6}{Comparing polarized detection
  microscope designs. Each row in this figure corresponds to a row in the middle
  third of Table 1.}{compare-illumination}
\fig{../figures/compare-both.pdf}{0.6}{Comparing polarized illumination and
  detection microscope designs. Each row in this figure corresponds to a row in
  the bottom third of Table 1. The illumination and detection polarization
  arrows in the schematic are offset by a few degrees for visualization purposes
  only---in the simulations the orientations are
  identical.}{compare-both}

\subsection{Minimum Number of Illumination Polarizations}
In the paper draft we only considered microscopes with four illumination
polarizations per view. Figure \ref{fig:compare-illumination} shows a comparison
of several polarized illumination microscopes and the top third of Table 1 shows
the summary statistics.

We find that for low-NA illumination with a single view there is no advantage to
using more than three polarization orientations. Two polarization orientations
is too few because the system suffers from an extra symmetry. Four polarization
orientations adds no extra information, and in a real system each polarized
illumination orientation requires extra switching and data processing time.

We also find that for low-NA illumination with two views that two polarization
orientations per view is sufficient for an orientation reconstruction.

Recall that these results use a Poisson noise model. I expect that with a
Poisson+Gaussian model that fewer polarization frames would be even more
advantageous because each intensity measurement will collect more photons and
each measurement would avoid the noise floor.

Figures \ref{fig:ill-pol} and \ref{fig:ill-pol-ortho} show how the number of
polarization frames affects the reconstruction when the illumination NA is
larger than 0. For large illumination NA, more polarized illumination
orientations offers slightly better reconstructions. As the illumination NA
decreases, the benefit of increasing the number of polarization orientations
vanishes.

\fig{../figures/ill-polarization.pdf}{1.0}{Single-view K\"ohler illumination
  microscope with varying number of illumination polarization orientations and
  illumination NA. b) Solid-angle uncertainty for the microscope in a). c)
  Median of the solid-angle uncertainty as a function of illumination
  polarization orientation and illumination NA. d) MAD of the solid-angle
  uncertainty as a function of illumination polarization orientation and
  illumination NA. The microscope in a) and b) is indicated by a cross in c) and
  d).}{ill-pol}

\fig{../figures/ill-polarization-ortho.pdf}{1.0}{Dual-view symmetric K\"ohler
  illumination microscope with varying number of illumination polarization
  orientations and illumination NA. b) Solid-angle uncertainty for the
  microscope in a). c) Median of the solid-angle uncertainty as a function of
  illumination polarization orientation and illumination NA. d) MAD of the
  solid-angle uncertainty as a function of illumination polarization orientation
  and illumination NA. The microscope in a) and b) is indicated by a cross in c)
  and d).}{ill-pol-ortho}

\subsection{Polarized Illumination vs. Polarized Detection}
In the paper we only considered polarized illumination microscopes and ignored
polarized detection microscopes. Figure \ref{fig:compare-illumination} shows a
comparison of several polarized detection microscopes and the middle third of
Table 1 shows the summary statistics.

Polarized illumination microscopes outperform polarized detection microscopes
with the same number of frames. The main difference between polarized detection
and polarized illumination is the size of the intensity signals for the same
sample exposure. Polarized detection signals are smaller than polarized
illumination signals because the polarizer blocks half of the photons from the
detector, so polarized detection signals suffer from a relative uncertainty.

The preference for polarized illumination would be even stronger if we used a
Poisson+Gaussian noise model because larger intensity measurements would contain
more information about the fluorophore orientation.

Note that these results only apply if we split the sample exposure equally
between intensity measurements. If a lossless beam splitter is available, then
adding the beam splitter is a ``free'' source of information about the
fluorophore orientation.

\section{Polarized Illumination and Detection}
Figure \ref{fig:compare-both} shows a comparison of several polarized
illumination+detection microscopes, and the bottom third of Table
1 shows the summary statistics.

Polarized illumination outperforms the polarized illumination+detection
microscopes. Some of the polarizer orientation combinations do not give much
information about the fluorophore orientation, so they are relatively useless
measurements. This results shows that more measurements are not always
beneficial.

Once again, these results only apply if we split the sample exposure equally
between intensity measurements. Adding a lossless beam splitter is still a good
choice if you have the option.

\begin{table}[ht!]
\centering
\caption{Comparison of designs in each class of microscope. All solid-angle
  uncertainty statistics are in steradians. }
\begin{tabular}{lllllll}
  \toprule
  Polarized Ill.&Number of &Polarization Settings&$n$-fold&Max\{$\sigma_{\Omega}$\}&Median\{$\sigma_{\Omega}$\}&MAD\{$\sigma_{\Omega}$\}\\ 
  or Det.&Views&Per View&Degeneracy&&&\\\midrule
  Illumination&1&2&8&5.10$\times 10^{00}$&1.65$\times 10^{-3}$&3.77$\times 10^{-4}$ \\ 
                &&3&4&5.10$\times 10^{00}$&1.65$\times 10^{-3}$&3.79$\times 10^{-4}$ \\
                &&4&4&5.10$\times 10^{00}$&1.65$\times 10^{-3}$&3.77$\times 10^{-4}$ \\\cline{2-7}  
                &2&2&2&6.23$\times 10^{-3}$&1.79$\times 10^{-3}$&1.66$\times 10^{-4}$\\
                &&3&2&1.24$\times 10^{-2}$&1.79$\times 10^{-3}$&1.66$\times 10^{-4}$\\  
                &&4&2&6.23$\times 10^{-3}$&1.79$\times 10^{-3}$&1.65$\times 10^{-4}$\\\hline  
  Detection&1&2&8&9.16$\times 10^{+2}$&2.62$\times 10^{-3}$&1.08$\times 10^{-3}$\\
                &&3&4&1.30$\times 10^{+1}$&2.38$\times 10^{-3}$&7.62$\times 10^{-4}$\\
                &&4&4&1.30$\times 10^{+1}$&2.28$\times 10^{-3}$&6.55$\times 10^{-4}$\\\cline{2-7}      
                &2&2&8&2.41$\times 10^{+1}$&3.20$\times 10^{-3}$&8.01$\times 10^{-4}$\\
                &&3&2&8.02$\times 10^{-2}$&2.97$\times 10^{-3}$&7.32$\times 10^{-4}$\\  
                &&4&2&8.08$\times 10^{-2}$&3.06$\times 10^{-3}$&7.96$\times 10^{-4}$\\\hline
  Both&1&2 ill. $\times$ 2 det. = 4&8&7.22$\times 10^{00}$&2.66$\times 10^{-3}$&6.63$\times 10^{-4}$\\
                &&3 ill. $\times$ 3 det. = 9&4&7.21$\times 10^{00}$&2.63$\times 10^{-3}$&6.19$\times 10^{-4}$\\
                &&2 ill. $\times$ 4 det. = 8&4&7.19$\times 10^{00}$&2.62$\times 10^{-3}$&6.04$\times 10^{-4}$\\\cline{2-7}      
                &2&2 ill. $\times$ 2 det. = 4&2&7.04$\times 10^{-3}$&2.67$\times 10^{-3}$&3.76$\times 10^{-4}$\\
                &&3 ill. $\times$ 3 det. = 9&2&1.72$\times 10^{-2}$&2.66$\times 10^{-3}$&4.84$\times 10^{-4}$\\
                &&2 ill. $\times$ 4 det. = 8&2&7.49$\times 10^{-3}$&2.70$\times 10^{-3}$&5.37$\times 10^{-4}$\\    
  \bottomrule
\end{tabular}
\end{table}


\end{document}

