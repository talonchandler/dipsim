\documentclass[11pt]{article}

%%%%%%%%%%%%
% Packages %
%%%%%%%%%%%%
\hyphenpenalty=10000
\usepackage{tocloft}
\renewcommand\cftsecleader{\cftdotfill{\cftdotsep}}
\def\undertilde#1{\mathord{\vtop{\ialign{##\crcr
$\hfil\displaystyle{#1}\hfil$\crcr\noalign{\kern1.5pt\nointerlineskip}
$\hfil\tilde{}\hfil$\crcr\noalign{\kern1.5pt}}}}}
\usepackage{cleveref}
\usepackage{xcolor}
\usepackage{hyperref}
\usepackage{epstopdf}
\usepackage{braket}
\usepackage{upgreek}
\usepackage{caption}
\usepackage{booktabs}
\usepackage{subcaption}
\usepackage{amssymb,latexsym,amsmath,gensymb}
\usepackage{latexsym}
\usepackage{graphicx}
\usepackage{float}
\usepackage{enumitem}
\usepackage{pdflscape}
\usepackage{url}
\usepackage{tikz, calc}
\usetikzlibrary{shapes.geometric, arrows, calc}
\tikzstyle{norm} = [rectangle, rounded corners, minimum width=2cm, minimum height=1cm,text centered, draw=black]
\tikzstyle{arrow} = [thick, ->, >=stealth]

\providecommand{\e}[1]{\ensuremath{\times 10^{#1}}} 
\providecommand{\mb}[1]{\mathbf{#1}}
\providecommand{\mh}[1]{\mathbf{\hat{#1}}}
\providecommand{\bs}[1]{\boldsymbol{#1}} 
\providecommand{\intinf}{\int_{-\infty}^{\infty}}
\providecommand{\fig}[4]{
  % filename, width, caption, label
\begin{figure}[h]
 \captionsetup{width=1.0\linewidth}
 \centering
 \includegraphics[width = #2\textwidth]{#1}
 \caption{#3}
 \label{fig:#4}
\end{figure}
}

\newcommand{\tensor}[1]{\overset{\text{\tiny$\leftrightarrow$}}{\mb{#1}}}
\newcommand{\tunderbrace}[2]{\underbrace{#1}_{\textstyle#2}}
\providecommand{\figs}[7]{
  % filename1, filename2, caption1, caption2, label1, label2, shift
\begin{figure}[H]
\centering
\begin{minipage}[b]{.45\textwidth}
  \centering
  \includegraphics[width=1.0\linewidth]{#1}
  \captionsetup{justification=justified, singlelinecheck=true}
  \caption{#3}
  \label{fig:#5}
\end{minipage}
\hspace{2em}
\begin{minipage}[b]{.45\textwidth}
  \centering
  \includegraphics[width=1.0\linewidth]{#2}
  \vspace{#7em}
  \captionsetup{justification=justified}
  \caption{#4}
  \label{fig:#6}
\end{minipage}
\end{figure}
}
\makeatletter

\providecommand{\code}[1]{
\begin{center}
\lstinputlisting{#1}
\end{center}
}

\newcommand{\crefrangeconjunction}{--}
%%%%%%%%%%%
% Spacing %
%%%%%%%%%%%
% Margins
\usepackage[
top    = 1.5cm,
bottom = 1.5cm,
left   = 1.5cm,
right  = 1.5cm]{geometry}

% Indents, paragraph space
%\usepackage{parskip}
\setlength{\parskip}{1.5ex}

% Section spacing
\usepackage{titlesec}
\titlespacing*{\title}
{0pt}{0ex}{0ex}
\titlespacing*{\section}
{0pt}{0ex}{0ex}
\titlespacing*{\subsection}
{0pt}{0ex}{0ex}
\titlespacing*{\subsubsection}
{0pt}{0ex}{0ex}

% Line spacing
\linespread{1.1}

%%%%%%%%%%%%
% Document %
%%%%%%%%%%%%
\begin{document}
\title{\vspace{-2.5em} Verifying The Polarized Fluorescence Microscopy Model Using Anisotropy \vspace{-1em}} \author{Talon
  Chandler}% and Patrick La Rivi\`ere}
\date{\vspace{-1em}\today\vspace{-1em}}
\maketitle
\section{Introduction}
Chemists often use the \textit{anisotropy} of a sample to characterize
rotational dynamics and the transition moments of molecules. In these notes I
will use the polarized fluorescence microscopy model from previous notes to
calculate the anisotropy of several fluorophore distributions. I will reproduce
known results to verify the model. This work is useful for verifying that our
model is correct and for relating our work to published results.

\section{Measuring The Anisotropy}
The anisotropy, $r$, of a sample is measured by (1) exciting a sample with a
low-NA beam of linearly polarized light, (2) placing a low-NA detection arm
orthogonal to the illumination arm, (3) placing a linear polarizer in the
detection arm, (4) measuring the intensity with the detection polarizer parallel
($I_\parallel$) and perpendicular ($I_\perp$) to the illumination polarization,
(5) calculating the anisotropy using
\begin{align}
  r = \frac{I_\parallel - I_\perp}{I_\parallel + 2I_\perp}.
\end{align}
See Figure 1 for a schematic of the $I_\parallel$ and $I_\perp$ measurements.
\fig{../figures/schematic.pdf}{0.7}{Schematics for anisotropy
  measurements. Solid (dashed) arrows show the illumination (detection)
  polarization orientation.
  $\text{NA}_{\text{ill}} = \text{NA}_{\text{det}} = 0$.}{schematic}



Lakowicz \cite{lak} gives several known anisotropies for special dipole
distributions. We assume that the excitation and emission dipole moments are
collinear. If the dipole moment orientations are
\begin{itemize}
\item completely aligned and parallel to the excitation polarization
  $\rightarrow r = 1$
\item completely aligned and perpendicular to the excitation polarization
  $\rightarrow r = -0.5$
\item uniformly distributed $\rightarrow r=0.4$.
\end{itemize}

\section{Calculating The Anisotropy With The Polarized Fluorescence Microscopy Model}
We can calculate the intensities measured by the experiments in Figure 1 using
\begin{align}
  I \propto \int_{\mathbb{S}^2}d\mh{r} f(\mh{r}; \hat{\bs{\mu}}, \kappa) \eta_{\text{exc}}(\mh{r}) \eta_{\text{det}}(\mh{r}) \label{eq:ensemble_det}%
\end{align}
where $f(\mh{r}; \hat{\bs{\mu}}, \kappa)$ is the Watson distribution with
central orientation $\hat{\bs{\mu}}$ and concentration parameter $\kappa$,
$\eta_{\text{exc}}$ is the excitation efficiency of a single fluorophore and
$\eta_{\text{det}}$ is the detection efficiency of a single fluorophore. See
previous note sets and the paper for the efficiency expressions. 

Figures \ref{fig:results} and \ref{fig:profile} show the intensities and
anisotropy of the experiment in Figure 1 as a function of fluorophore
distribution ($\hat{\bs{\mu}}$ and $\kappa$). The results match the special
cases mentioned in Lakowicz.  When the fluorophores are completely aligned
($\kappa = \infty$) and parallel to the excitation polarization ($\mh{y}$-axis)
then $r = 1$ (see bottom right sphere in Figure \ref{fig:results} or right plot
in Figure \ref{fig:profile}). When the fluorophores are completely aligned
($\kappa = \infty$) and perpendicular to the excitation polarization
($\mh{x}$-axis) then $r = -0.5$ (see bottom right sphere in Figure
\ref{fig:results} or right plot in Figure \ref{fig:profile}). Finally, when the
fluorophores are uniformly distributed ($\kappa = 0$) then $r = 0.4$ (see Figure
\ref{fig:results} bottom row second from the left or the right plot in Figure
\ref{fig:profile}).

\fig{../figures/anisotropy-kappa.pdf}{1.0}{Intensities and anisotropy as a
  function of fluorophore distribution. \textbf{Rows:} 1) $I_\perp$ see Figure
  1; 2) $I_\parallel$ see Figure 1; 3) Anisotropy see Equation
  1. \textbf{Columns:} Varying concentration parameter of the Watson
  distribution. Note that $\kappa = \infty$ corresponds to a single fluorophore
  or a perfectly concentrated ensemble of fluorophores.}{results}

\fig{../figures/anisotropy-profile.pdf}{1.0}{Profiles of the spheres in Figure 2
  in the $x-y$ plane. $\phi$ is the azimuth angle measured from the $+x$-axis in
  the $x-y$ plane. \textbf{Columns:} 1) $I_\perp$ see Figure 1; 2) $I_\parallel$
  see Figure 1; 3) Anisotropy see Equation 1. \textbf{Colors:} Varying
  concentration parameter of the Watson distribution.}{profile}

\bibliography{report}{}
\bibliographystyle{unsrt}

\end{document}

